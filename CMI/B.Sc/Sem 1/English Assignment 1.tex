\documentclass{article}
\usepackage{fullpage}
\usepackage{amsmath}
\usepackage{amsfonts}
\usepackage{authblk}
\usepackage{titling}
\usepackage{tikz}

\title{\huge{English Assignment\\\hspace{7cm}- Usha Mahadevan}
}
\author{Soham Chatterjee\\Roll: BMC202175}
\date{}
\setlength{\parindent}{0cm}
\renewcommand\maketitlehooka{\null\mbox{}\vfill}
\renewcommand\maketitlehookd{\vfill\null}
\begin{document}
	\textbf{Question:} Irony is R.K. Narayan’s forte. Examine the element of irony in ‘An Astrologer’s Day’
	
	\textbf{Answer:} $$`\text{Irony is R.K. Narayan’s forte}'$$We will see that the story `An Astrologer's Day' do has some instances which can be considered as an irony.
	
	\hspace*{1cm}In the first two paragraphs we can see the astrologer dresses and makes his appearance in such a way that his customers believes him as a true astrologer without any doubt, even though``{He was as much a stranger to the stars as were his innocent customers.}" He pleases all his customers with a ``matter of study, practice, and shrewd guesswork." Which implies that he is dishonest and cheating on his customers. Here, by saying ``{All the same, it was as much an honest man’s labour as any other, and he deserved the wages he carried home at the end of a day"} the author makes an irony.
	
	\hspace*{1cm}We can see another instance of irony when the astrologer meets Guru Nayak. In the story we see that once the astrologer tried to kill Guru Nayak. So it was clear that he was seeking revenge from the astrologer. Hence the astrologer's life was in danger. It is ironical that he claims himself as an astrologer but he is completely unaware of his own future.
	
	\hspace*{1cm}Guru Nayak asked the astrologer about the person he was looking for to get some clue about the person for his revenge. Ironically the astrologer himself was that person. Guru Nayak pair him the money without knowing that it was the person whom he paid he was looking for.
	
	\hspace*{1cm}These were the instances in the story `An Astrologer's Day' where we can see irony. Clearly it is indeed true that `Irony is R.K. Narayan’s forte'.
\end{document}