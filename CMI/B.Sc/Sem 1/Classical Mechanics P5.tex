\documentclass{article}
\usepackage{fullpage}
\usepackage{amsmath}
\usepackage{amsfonts}
\usepackage{authblk}
\usepackage{titling}
\usepackage{tikz}

\title{\huge{Classical Mechanics 1, Autumn 2021 CMI \\ Problem set 5\\\hspace{7cm}- Govind S. Krishnaswami}
}
\author{Soham Chatterjee\\Roll: BMC202175}
\date{}

\newcommand{\bma}{\boldsymbol{a}}
\newcommand{\bmb}{\boldsymbol{b}}
\newcommand{\bmc}{\boldsymbol{c}}
\newcommand{\bmr}{\boldsymbol{r}}
\newcommand{\bmv}{\boldsymbol{v}}
\newcommand{\bmp}{\boldsymbol{p}}
\newcommand{\bmF}{\boldsymbol{F}}
\newcommand{\bmf}{\boldsymbol{f}}
\renewcommand\maketitlehooka{\null\mbox{}\vfill}
\renewcommand\maketitlehookd{\vfill\null}

\setlength{\parindent}{1cm}
\begin{document}
	\maketitle\pagebreak
\begin{enumerate}
	\item \setlength{\parindent}{1cm}Given that the particle $A$ of mass $m_a$ exerts a force $F_b$ on particle $B$ of mass $m_b$ and the particle $B$ exerts a force $F_a$ on particle $A$. The position vector of particle $A$ is $\bmr_{\bma}$ and position vector of particle $B$ is $\bmr_{\bmb}$. 
	
	Hence $$F_a=m_a\ddot{\bmr}_{\bma}\text{ and }F_b=m_b\ddot{\bmr}_{\bmb}$$Now in this two particle system there are no external force acting. Therefore the net internal force must be 0. Hence $$F_a+F_b=0\implies m_a\ddot{\bmr}_{\bma}+m_b\ddot{\bmr}_{\bmb}=0$$Hence the Newton’s 2nd law equations of motion for this system in an inertial frame where the position vectors of the particles are $\bmr_{\bma}$ and $\bmr_{\bmb}$ is $$m_a\ddot{\bmr}_{\bma}+m_b\ddot{\bmr}_{\bmb}=0$$Now we can say $$m_a\ddot{\bmr}_{\bma}+m_b\ddot{\bmr}_{\bmb}=0\implies \frac{d}{dt}\left(m_a\dot{\bmr}_{\bma}+m_b\dot{\bmr}_{\bmb}\right)=0\implies m_a\dot{\bmr}_{\bma}+m_b\dot{\bmr}_{\bmb}=\text{constant} $$Now $m_a\dot{\bmr}_{\bma}+m_b\dot{\bmr}_{\bmb}$ is the total momentum of the system. Hence the total momentum of the system is constant. Or we can say the total momentum of the system is independent of time.

	
	\item No, the events will not occur at the same location with respect to all other inertial frames that are in uniform motion relative to $S$.
	
	Suppose the two frames coincide at $t=0$. The position vector of the two events with respect to  frame $S$ is $\bmr_{\boldsymbol{0}}$. Let the other frame is moving with uniform velocity $\bmv$ relative to the frame $S$. Hence with respect to the other frame at time $t=t_A$ the event $A$ occur at the location $\bmr_{\boldsymbol{A}}=\bmr_{\boldsymbol{0}}+\bmv t_A$ and at time $t=t_B$ the event $B$ occur at the location $\bmr_{\boldsymbol{B}}=\bmr_{\boldsymbol{0}}+\bmv t_B$.
		
		Therefore the difference of the location of occurrence of events $A$ and $B$ with respect to the other frame is $$\bmr_{\boldsymbol{B}}-\bmr_{\boldsymbol{A}}=\bmr_{\boldsymbol{0}}+\bmv t_B-(\bmr_{\boldsymbol{0}}+\bmv t_A)=\bmv(t_B-t_A)$$Since given that $t_B>t_A$ hence $\bmv(t_B-t_A)\neq 0$. Therefore the events occur at different location with respect to the other frame.
	
	\item Yes, the events $A$ and $B$ occur simultaneously a distance $d$ apart in all other inertial frames that are in uniform motion relative to $S$.
	
	Suppose the two frames coincide at $t=0$. The position vector of the two events with respect to  frame $S$ is $\bmr_{\boldsymbol{0}}$. Let the other frame is moving with uniform velocity $\bmv$ relative to the frame $S$. Let the position vectors of the location of the events $A$ and $A$ are respectively $\bmr_{\bma}$ and $\bmr_{\bmb}$. Hence according to the question $$|\bmr_{\bma}-\bmr_{\bmb}|=d$$
	
	Let the events occur at time $t=T$. Then with respect to the other frame the location of the events $A$ and $B$ are respectively $\bmr_{\bma}'=\bmr_{\bma}+\bmv T$ and $\bmr_{\bmb}'=\bmr_{\bmb}+\bmv T$. Hence the distance between the locations of events $A$ and $B$ with respect to the other frame is $$|\bmr_{\bma}'-\bmr_{\bmb}'|=|\bmr_{\bma}+\bmv T-(\bmr_{\bmb}+\bmv T)|=|\bmr_{\bma}-\bmr_{\bmb}|=d$$
	
	Hence the events $A$ and $B$ occur simultaneously a distance $d$ apart in all other inertial frames that are in uniform motion relative to $S$.
	
	\item In the CM Problem Set 4 P4 we found that \begin{align*}
		&A^{2n}=(A^2)^n=\left( -\frac{k}{m}I\right)^n=(-1)^n\frac{k^n}{m^n}I\\
		&A^{2n+1}=A^{2n}A=\left((-1)^n\frac{k^n}{m^n}I \right) A=(-1)^n\frac{k^n}{m^n}A=(-1)^n\frac{k^n}{m^n}\begin{pmatrix}
			0&\frac1m\\
			-k&0
		\end{pmatrix}
	\end{align*}Now\begin{align*}
	e^{At}\ &=\sum\limits_{p=0}^{\infty}\frac{(At)^p}{p!}\\
	&=\sum\limits_{p=0}^{\infty}\frac{(At)^{2p}}{(2p)!}+\sum\limits_{p=0}^{\infty}\frac{(At)^{2p+1}}{(2p+1)!}\\
	&=\sum\limits_{p=0}^{\infty}(-1)^p\frac{k^p}{m^p}I\frac{t^{2p}}{(2p)!}+\sum\limits_{p=0}^{\infty}(-1)^p\frac{k^p}{m^p}\begin{pmatrix}
		0&\frac1m\\
		-k&0
	\end{pmatrix}\frac{t^{2p+1}}{(2p+1)!}\\
	&=\Bigg[\sum\limits_{p=0}^{\infty}(-1)^p\frac{k^p}{m^p}\frac{t^{2p}}{(2p)!}\Bigg]I+\Bigg[\sum\limits_{p=0}^{\infty}(-1)^p\frac{k^p}{m^p}\frac{t^{2p+1}}{(2p+1)!}\Bigg]\begin{pmatrix}
		0&\frac1m\\
		-k&0
	\end{pmatrix}\\
&=\Bigg[\sum\limits_{p=0}^{\infty}(-1)^p\left( \sqrt{\frac{k}{m}}\right) ^{2p}\frac{t^{2p}}{(2p)!}\Bigg]I+\Bigg[ \sqrt{\frac{m}{k}}\sum\limits_{p=0}^{\infty}(-1)^p\left( \sqrt{\frac{k}{m}}\right)^{2p+1}\frac{t^{2p+1}}{(2p+1)!}\Bigg]\begin{pmatrix}
	0&\frac1m\\
	-k&0
\end{pmatrix}\\
&=\cos\left( \sqrt{\frac{k}{m}}t \right) I+ \sqrt{\frac{m}{k}}\sin\left( \sqrt{\frac{k}{m}}t \right) \begin{pmatrix}
	0&\frac1m\\
	-k&0
\end{pmatrix}\\
&=\cos\left( \omega t \right) I+\frac1{\omega} \sin\left( \omega t \right)\begin{pmatrix}
	0&\frac1m\\
	-k&0
\end{pmatrix}\\
&=\cos\left( \omega t \right) I+ \sin\left( \omega t \right)\begin{pmatrix}
	0&\frac1{m\omega}\\
	-\frac{k}{\omega}&0
\end{pmatrix}\\
&=\begin{pmatrix}
	\cos\left( \omega t \right) &0\\
	0&\cos\left( \omega t \right) 
\end{pmatrix}+\begin{pmatrix}
	0&\frac1{m\omega}\sin\left( \omega t \right)\\
	-\frac{k}{\omega}\sin\left(\omega t \right)&0
\end{pmatrix}\\
&=\begin{pmatrix}
	\cos\left(\omega t \right) &\frac1{m\omega}\sin\left( \omega t \right)\\
	-\frac{k}{\omega}\sin\left(\omega t \right)&	\cos\left( \omega t \right) 
\end{pmatrix}
\end{align*}
Therefore$$e^{At}=\begin{pmatrix}
\cos\left(\omega t \right) &\frac1{m\omega}\sin\left( \omega t \right)\\
-\frac{k}{\omega}\sin\left(\omega t \right)&	\cos\left( \omega t \right) 
\end{pmatrix}$$
\end{enumerate}
\end{document}