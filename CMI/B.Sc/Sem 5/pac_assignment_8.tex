\documentclass[a4paper, 11pt]{article}
\usepackage{comment} % enables the use of multi-line comments (\ifx \fi) 
\usepackage{fullpage} % changes the margin
\usepackage[a4paper, total={7in, 10in}]{geometry}
\usepackage{amsmath,mathtools,mathdots,amssymb}
\usepackage{amssymb,amsthm}  % assumes amsmath package installed
\usepackage{float}
\usepackage{xcolor}
\usepackage{mdframed}
\usepackage[shortlabels]{enumitem}
\usepackage{indentfirst}
\usepackage{hyperref}
\hypersetup{
	colorlinks=true,
	linkcolor=blue,
	filecolor=magenta,      
	urlcolor=blue!70!red,
	pdftitle={Assignment}, %%%%%%%%%%%%%%%%   WRITE ASSIGNMENT PDF NAME  %%%%%%%%%%%%%%%%%%%%
}
\usepackage[most,many,breakable]{tcolorbox}
\usepackage{tikz}
\usetikzlibrary{decorations.pathreplacing,angles,quotes,patterns}
\usetikzlibrary{decorations.shapes}
\usepackage{caption}
\usepackage{mathpazo}


\definecolor{mytheorembg}{HTML}{F2F2F9}
\definecolor{mytheoremfr}{HTML}{00007B}


\tcbuselibrary{theorems,skins,hooks}
\newtcbtheorem{problem}{Problem}
{%
	enhanced,
	breakable,
	colback = mytheorembg,
	frame hidden,
	boxrule = 0sp,
	borderline west = {2pt}{0pt}{mytheoremfr},
	sharp corners,
	detach title,
	before upper = \tcbtitle\par\smallskip,
	coltitle = mytheoremfr,
	fonttitle = \bfseries\sffamily,
	description font = \mdseries,
	separator sign none,
	segmentation style={solid, mytheoremfr},
}
{p}
\usepackage[ruled,vlined,linesnumbered]{algorithm2e}
% To give references for any problem use like this
% suppose the problem number is p3 then 2 options either 
% \hyperref[p:p3]{<text you want to use to hyperlink> \ref{p:p3}}
%                  or directly 
%                   \ref{p:p3}

\tikzset{decorate sep/.style 2 args=
	{decorate,decoration={shape backgrounds,shape=circle,shape size=#1,shape sep=#2}}}

%---------------------------------------
% BlackBoard Math Fonts :-
%---------------------------------------

%Captital Letters
\newcommand{\bbA}{\mathbb{A}}	\newcommand{\bbB}{\mathbb{B}}
\newcommand{\bbC}{\mathbb{C}}	\newcommand{\bbD}{\mathbb{D}}
\newcommand{\bbE}{\mathbb{E}}	\newcommand{\bbF}{\mathbb{F}}
\newcommand{\bbG}{\mathbb{G}}	\newcommand{\bbH}{\mathbb{H}}
\newcommand{\bbI}{\mathbb{I}}	\newcommand{\bbJ}{\mathbb{J}}
\newcommand{\bbK}{\mathbb{K}}	\newcommand{\bbL}{\mathbb{L}}
\newcommand{\bbM}{\mathbb{M}}	\newcommand{\bbN}{\mathbb{N}}
\newcommand{\bbO}{\mathbb{O}}	\newcommand{\bbP}{\mathbb{P}}
\newcommand{\bbQ}{\mathbb{Q}}	\newcommand{\bbR}{\mathbb{R}}
\newcommand{\bbS}{\mathbb{S}}	\newcommand{\bbT}{\mathbb{T}}
\newcommand{\bbU}{\mathbb{U}}	\newcommand{\bbV}{\mathbb{V}}
\newcommand{\bbW}{\mathbb{W}}	\newcommand{\bbX}{\mathbb{X}}
\newcommand{\bbY}{\mathbb{Y}}	\newcommand{\bbZ}{\mathbb{Z}}

%---------------------------------------
% MathCal Fonts :-
%---------------------------------------

%Captital Letters
\newcommand{\mcA}{\mathcal{A}}	\newcommand{\mcB}{\mathcal{B}}
\newcommand{\mcC}{\mathcal{C}}	\newcommand{\mcD}{\mathcal{D}}
\newcommand{\mcE}{\mathcal{E}}	\newcommand{\mcF}{\mathcal{F}}
\newcommand{\mcG}{\mathcal{G}}	\newcommand{\mcH}{\mathcal{H}}
\newcommand{\mcI}{\mathcal{I}}	\newcommand{\mcJ}{\mathcal{J}}
\newcommand{\mcK}{\mathcal{K}}	\newcommand{\mcL}{\mathcal{L}}
\newcommand{\mcM}{\mathcal{M}}	\newcommand{\mcN}{\mathcal{N}}
\newcommand{\mcO}{\mathcal{O}}	\newcommand{\mcP}{\mathcal{P}}
\newcommand{\mcQ}{\mathcal{Q}}	\newcommand{\mcR}{\mathcal{R}}
\newcommand{\mcS}{\mathcal{S}}	\newcommand{\mcT}{\mathcal{T}}
\newcommand{\mcU}{\mathcal{U}}	\newcommand{\mcV}{\mathcal{V}}
\newcommand{\mcW}{\mathcal{W}}	\newcommand{\mcX}{\mathcal{X}}
\newcommand{\mcY}{\mathcal{Y}}	\newcommand{\mcZ}{\mathcal{Z}}



%---------------------------------------
% Bold Math Fonts :-
%---------------------------------------

%Captital Letters
\newcommand{\bmA}{\boldsymbol{A}}	\newcommand{\bmB}{\boldsymbol{B}}
\newcommand{\bmC}{\boldsymbol{C}}	\newcommand{\bmD}{\boldsymbol{D}}
\newcommand{\bmE}{\boldsymbol{E}}	\newcommand{\bmF}{\boldsymbol{F}}
\newcommand{\bmG}{\boldsymbol{G}}	\newcommand{\bmH}{\boldsymbol{H}}
\newcommand{\bmI}{\boldsymbol{I}}	\newcommand{\bmJ}{\boldsymbol{J}}
\newcommand{\bmK}{\boldsymbol{K}}	\newcommand{\bmL}{\boldsymbol{L}}
\newcommand{\bmM}{\boldsymbol{M}}	\newcommand{\bmN}{\boldsymbol{N}}
\newcommand{\bmO}{\boldsymbol{O}}	\newcommand{\bmP}{\boldsymbol{P}}
\newcommand{\bmQ}{\boldsymbol{Q}}	\newcommand{\bmR}{\boldsymbol{R}}
\newcommand{\bmS}{\boldsymbol{S}}	\newcommand{\bmT}{\boldsymbol{T}}
\newcommand{\bmU}{\boldsymbol{U}}	\newcommand{\bmV}{\boldsymbol{V}}
\newcommand{\bmW}{\boldsymbol{W}}	\newcommand{\bmX}{\boldsymbol{X}}
\newcommand{\bmY}{\boldsymbol{Y}}	\newcommand{\bmZ}{\boldsymbol{Z}}
%Small Letters
\newcommand{\bma}{\boldsymbol{a}}	\newcommand{\bmb}{\boldsymbol{b}}
\newcommand{\bmc}{\boldsymbol{c}}	\newcommand{\bmd}{\boldsymbol{d}}
\newcommand{\bme}{\boldsymbol{e}}	\newcommand{\bmf}{\boldsymbol{f}}
\newcommand{\bmg}{\boldsymbol{g}}	\newcommand{\bmh}{\boldsymbol{h}}
\newcommand{\bmi}{\boldsymbol{i}}	\newcommand{\bmj}{\boldsymbol{j}}
\newcommand{\bmk}{\boldsymbol{k}}	\newcommand{\bml}{\boldsymbol{l}}
\newcommand{\bmm}{\boldsymbol{m}}	\newcommand{\bmn}{\boldsymbol{n}}
\newcommand{\bmo}{\boldsymbol{o}}	\newcommand{\bmp}{\boldsymbol{p}}
\newcommand{\bmq}{\boldsymbol{q}}	\newcommand{\bmr}{\boldsymbol{r}}
\newcommand{\bms}{\boldsymbol{s}}	\newcommand{\bmt}{\boldsymbol{t}}
\newcommand{\bmu}{\boldsymbol{u}}	\newcommand{\bmv}{\boldsymbol{v}}
\newcommand{\bmw}{\boldsymbol{w}}	\newcommand{\bmx}{\boldsymbol{x}}
\newcommand{\bmy}{\boldsymbol{y}}	\newcommand{\bmz}{\boldsymbol{z}}

%---------------------------------------
% Scr Math Fonts :-
%---------------------------------------

\newcommand{\sA}{{\mathscr{A}}}   \newcommand{\sB}{{\mathscr{B}}}
\newcommand{\sC}{{\mathscr{C}}}   \newcommand{\sD}{{\mathscr{D}}}
\newcommand{\sE}{{\mathscr{E}}}   \newcommand{\sF}{{\mathscr{F}}}
\newcommand{\sG}{{\mathscr{G}}}   \newcommand{\sH}{{\mathscr{H}}}
\newcommand{\sI}{{\mathscr{I}}}   \newcommand{\sJ}{{\mathscr{J}}}
\newcommand{\sK}{{\mathscr{K}}}   \newcommand{\sL}{{\mathscr{L}}}
\newcommand{\sM}{{\mathscr{M}}}   \newcommand{\sN}{{\mathscr{N}}}
\newcommand{\sO}{{\mathscr{O}}}   \newcommand{\sP}{{\mathscr{P}}}
\newcommand{\sQ}{{\mathscr{Q}}}   \newcommand{\sR}{{\mathscr{R}}}
\newcommand{\sS}{{\mathscr{S}}}   \newcommand{\sT}{{\mathscr{T}}}
\newcommand{\sU}{{\mathscr{U}}}   \newcommand{\sV}{{\mathscr{V}}}
\newcommand{\sW}{{\mathscr{W}}}   \newcommand{\sX}{{\mathscr{X}}}
\newcommand{\sY}{{\mathscr{Y}}}   \newcommand{\sZ}{{\mathscr{Z}}}


%---------------------------------------
% Math Fraktur Font
%---------------------------------------

%Captital Letters
\newcommand{\mfA}{\mathfrak{A}}	\newcommand{\mfB}{\mathfrak{B}}
\newcommand{\mfC}{\mathfrak{C}}	\newcommand{\mfD}{\mathfrak{D}}
\newcommand{\mfE}{\mathfrak{E}}	\newcommand{\mfF}{\mathfrak{F}}
\newcommand{\mfG}{\mathfrak{G}}	\newcommand{\mfH}{\mathfrak{H}}
\newcommand{\mfI}{\mathfrak{I}}	\newcommand{\mfJ}{\mathfrak{J}}
\newcommand{\mfK}{\mathfrak{K}}	\newcommand{\mfL}{\mathfrak{L}}
\newcommand{\mfM}{\mathfrak{M}}	\newcommand{\mfN}{\mathfrak{N}}
\newcommand{\mfO}{\mathfrak{O}}	\newcommand{\mfP}{\mathfrak{P}}
\newcommand{\mfQ}{\mathfrak{Q}}	\newcommand{\mfR}{\mathfrak{R}}
\newcommand{\mfS}{\mathfrak{S}}	\newcommand{\mfT}{\mathfrak{T}}
\newcommand{\mfU}{\mathfrak{U}}	\newcommand{\mfV}{\mathfrak{V}}
\newcommand{\mfW}{\mathfrak{W}}	\newcommand{\mfX}{\mathfrak{X}}
\newcommand{\mfY}{\mathfrak{Y}}	\newcommand{\mfZ}{\mathfrak{Z}}
%Small Letters
\newcommand{\mfa}{\mathfrak{a}}	\newcommand{\mfb}{\mathfrak{b}}
\newcommand{\mfc}{\mathfrak{c}}	\newcommand{\mfd}{\mathfrak{d}}
\newcommand{\mfe}{\mathfrak{e}}	\newcommand{\mff}{\mathfrak{f}}
\newcommand{\mfg}{\mathfrak{g}}	\newcommand{\mfh}{\mathfrak{h}}
\newcommand{\mfi}{\mathfrak{i}}	\newcommand{\mfj}{\mathfrak{j}}
\newcommand{\mfk}{\mathfrak{k}}	\newcommand{\mfl}{\mathfrak{l}}
\newcommand{\mfm}{\mathfrak{m}}	\newcommand{\mfn}{\mathfrak{n}}
\newcommand{\mfo}{\mathfrak{o}}	\newcommand{\mfp}{\mathfrak{p}}
\newcommand{\mfq}{\mathfrak{q}}	\newcommand{\mfr}{\mathfrak{r}}
\newcommand{\mfs}{\mathfrak{s}}	\newcommand{\mft}{\mathfrak{t}}
\newcommand{\mfu}{\mathfrak{u}}	\newcommand{\mfv}{\mathfrak{v}}
\newcommand{\mfw}{\mathfrak{w}}	\newcommand{\mfx}{\mathfrak{x}}
\newcommand{\mfy}{\mathfrak{y}}	\newcommand{\mfz}{\mathfrak{z}}

%---------------------------------------
% Bar
%---------------------------------------

%Captital Letters
\newcommand{\barA}{\overline{A}}	\newcommand{\barB}{\overline{B}}
\newcommand{\barC}{\overline{C}}	\newcommand{\barD}{\overline{D}}
\newcommand{\barE}{\overline{E}}	\newcommand{\barF}{\overline{F}}
\newcommand{\barG}{\overline{G}}	\newcommand{\barH}{\overline{H}}
\newcommand{\barI}{\overline{I}}	\newcommand{\barJ}{\overline{J}}
\newcommand{\barK}{\overline{K}}	\newcommand{\barL}{\overline{L}}
\newcommand{\barM}{\overline{M}}	\newcommand{\barN}{\overline{N}}
\newcommand{\barO}{\overline{O}}	\newcommand{\barP}{\overline{P}}
\newcommand{\barQ}{\overline{Q}}	\newcommand{\barR}{\overline{R}}
\newcommand{\barS}{\overline{S}}	\newcommand{\barT}{\overline{T}}
\newcommand{\barU}{\overline{U}}	\newcommand{\barV}{\overline{V}}
\newcommand{\barW}{\overline{W}}	\newcommand{\barX}{\overline{X}}
\newcommand{\barY}{\overline{Y}}	\newcommand{\barZ}{\overline{Z}}
%Small Letters
\newcommand{\bara}{\overline{a}}	\newcommand{\barb}{\overline{b}}
\newcommand{\barc}{\overline{c}}	\newcommand{\bard}{\overline{d}}
\newcommand{\bare}{\overline{e}}	\newcommand{\barf}{\overline{f}}
\newcommand{\barg}{\overline{g}}	\newcommand{\barh}{\overline{h}}
\newcommand{\bari}{\overline{i}}	\newcommand{\barj}{\overline{j}}
\newcommand{\bark}{\overline{k}}	\newcommand{\barl}{\overline{l}}
\newcommand{\barm}{\overline{m}}	\newcommand{\barn}{\overline{n}}
\newcommand{\baro}{\overline{o}}	\newcommand{\barp}{\overline{p}}
\newcommand{\barq}{\overline{q}}	\newcommand{\barr}{\overline{r}}
\newcommand{\bars}{\overline{s}}	\newcommand{\bart}{\overline{t}}
\newcommand{\baru}{\overline{u}}	\newcommand{\barv}{\overline{v}}
\newcommand{\barw}{\overline{w}}	\newcommand{\barx}{\overline{x}}
\newcommand{\bary}{\overline{y}}	\newcommand{\barz}{\overline{z}}

%---------------------------------------
% Greek Letters:-
%---------------------------------------
\newcommand{\eps}{\epsilon}
\newcommand{\veps}{\varepsilon}
\newcommand{\lm}{\lambda}
\newcommand{\Lm}{\Lambda}
\newcommand{\gm}{\gamma}
\newcommand{\Gm}{\Gamma}
\newcommand{\vph}{\varphi}
\newcommand{\ph}{\phi}

\newcommand{\Qed}{\begin{flushright}\qed\end{flushright}}
\newcommand{\parinn}{\setlength{\parindent}{1cm}}
\newcommand{\parinf}{\setlength{\parindent}{0cm}}
\newcommand{\del}[2]{\frac{\partial #1}{\partial #2}}
\newcommand{\Del}[3]{\frac{\partial^{#1} #2}{\partial^{#1} #3}}
\newcommand{\deld}[2]{\dfrac{\partial #1}{\partial #2}}
\newcommand{\Deld}[3]{\dfrac{\partial^{#1} #2}{\partial^{#1} #3}}
\newcommand{\uin}{\mathbin{\rotatebox[origin=c]{90}{$\in$}}}
\newcommand{\usubset}{\mathbin{\rotatebox[origin=c]{90}{$\subset$}}}
\newcommand{\lt}{\left}
\newcommand{\rt}{\right}
\newcommand{\exs}{\exists}
\newcommand{\st}{\strut}
\newcommand{\dps}[1]{\displaystyle{#1}}
\newcommand{\la}{\langle}
\newcommand{\ra}{\rangle}
\newenvironment{solution}
{\textit{\textbf{Solution:}} 
}
{ 
	\hfill $\blacksquare$
	
	\vspace{1cm}
}
\newcommand{\sol}[1]{\begin{solution}#1\end{solution}}
\newcommand{\solve}[1]{\setlength{\parindent}{0cm}\textbf{\textit{Solution: }}\setlength{\parindent}{1cm}#1 \Qed}
\newcommand{\mat}[1]{\left[\begin{matrix}#1\end{matrix}\right]}
\newcommand\numberthis{\addtocounter{equation}{1}\tag{\theequation}}
\newcommand{\handout}[3]{
	\noindent
	\begin{center}
		\framebox{
			\vbox{
				\hbox to 6.5in { {\bf Complexity Theory I } \hfill Jan -- May, 2023 }
				\vspace{4mm}
				\hbox to 6.5in { {\Large \hfill #1  \hfill} }
				\vspace{2mm}
				\hbox to 6.5in { {\em #2 \hfill #3} }
			}
		}
	\end{center}
	\vspace*{4mm}
}

\newcommand{\lecture}[3]{\handout{Lecture #1}{Lecturer: #2}{Scribe:	#3}}
	

\newcommand{\ov}[1]{\overline{#1}}
\newcommand{\thmref}[1]{\hyperref[#1]{Theorem \ref{#1}}}

\setlength{\parindent}{0pt}

%%%%%%%%%%%%%%%%%%%%%%%%%%%%%%%%%%%%%%%%%%%%%%%%%%%%%%%%%%%%%%%%%%%%%%%%%%%%%%%%%%%%%%%%%%%%%%%%%%%%%%%%%%%%%%%%%%%%%%%%%%%%%%%%%%%%%%%%

\begin{document}
	
%%%%%%%%%%%%%%%%%%%%%%%%%%%%%%%%%%%%%%%%%%%%%%%%%%%%%%%%%%%%%%%%%%%%%%%%%%%%%%%%%%%%%%%%%%%%%%%%%%%%%%%%%%%%%%%%%%%%%%%%%%%%%%%%%%%%%%%%
	
\textsf{\noindent \large\textbf{Soham Chatterjee} \hfill \textbf{Assignment - 8}\\
	Email: \href{sohamc@cmi.ac.in}{sohamc@cmi.ac.in} \hfill Roll: BMC202175\\
	\normalsize Course: Parallel Algorithms and Complexity \hfill Date: October 26, 2023}
	
%%%%%%%%%%%%%%%%%%%%%%%%%%%%%%%%%%%%%%%%%%%%%%%%%%%%%%%%%%%%%%%%%%%%%%%%%%%%%%%%%%%%%%%%%%%%%%%%%%%%%%%%%%%%%%%%%%%%%%%%%%%%%%%%%%%%%%%%
% Problem 1
%%%%%%%%%%%%%%%%%%%%%%%%%%%%%%%%%%%%%%%%%%%%%%%%%%%%%%%%%%%%%%%%%%%%%%%%%%%%%%%%%%%%%%%%%%%%%%%%%%%%%%%%%%%%%%%%%%%%%%%%%%%%%%%%%%%%%%%%
	
\begin{problem}{%problem statement
	}{p1% problem reference text
	}
$Pf'$ orientation of $C_1$ and $Pf'$ orientation of $C_2$ yield a $Pf'$ orientation of $C_1\oplus C_2$
		
%Problem		
\end{problem}
	
\solve{
	%Solution
	For any simple cycle $C$ let $f$ be the number of edges in forward direction and $k$ be the number of  vertices strictly inside $C$. Then we will show if for $C_1$ and $C_2$ we have $f_1,k_1$ and $f_2,k_2$ respectively and $f_1+k_1\equiv 1\pmod2$ and $f_2+k_2\equiv 1\pmod2$ then so is for the cycle $C\coloneqq C_1\oplus C_2$.
	
 Now let the path $P$ shared by $C_1$ and $C_2$ contains $m$ vertices. So the number of vertices of $C$ is $k=k_1+k_2+b$. Since all the edges of $P$ is either in forward direction with respect to $C_1$ or $C_2$ if we take $f_1+f_2$ then this number has all the edges of $P$. So the number of edges in forward direction of $C$ is $f=f_1+f_2-(b+1)$. Hence $$k+f=(k_1+k_2+b)+(f_1+f_2-(b+1))=(k_1+f_1)+(k_2+f_2)-1\equiv 1\pmod2$$So we obtain a $Pf'$ orientation of $C_1\oplus C_2$ 
}

	
%%%%%%%%%%%%%%%%%%%%%%%%%%%%%%%%%%%%%%%%%%%%%%%%%%%%%%%%%%%%%%%%%%%%%%%%%%%%%%%%%%%%%%%%%%%%%%%%%%%%%%%%%%%%%%%%%%%%%%%%%%%%%%%%%%%%%%%%
% Problem 2
%%%%%%%%%%%%%%%%%%%%%%%%%%%%%%%%%%%%%%%%%%%%%%%%%%%%%%%%%%%%%%%%%%%%%%%%%%%%%%%%%%%%%%%%%%%%%%%%%%%%%%%%%%%%%%%%%%%%%%%%%%%%%%%%%%%%%%%%

\begin{problem}{%problem statement
	}{p1% problem reference text
	}
Given any matrix of univariate polynomials of degree $\leq n^{O(1)}$ then prove that the coefficent of $x^i$ in the determinant of the matrix is in $GapL$
	%Problem		
\end{problem}

\solve{
	%Solution
	Suppose the given matrix is $M$. Now we replace the entries of $M$ with new variables. So the $(i,j)$th entry of $M$ is replaced by the variable $x_{i,j}$. Suppose the new matrix obtained is $M'$. Now using Mahajan-Vinay's method we obtain an arithmetic branching program which computes the determinant of $M'$. Now in the $ABP$ we replace every $x_{i,j}$ with the $(i,j)$th polynomial in $M$. So this new $ABP$ now computed the determinant of $M$. Let the source vertex of this is $s$ and target vertex is $t$. Let $\deg(\det M)=d$.
	
	We will do now homogenization of the $ABP$. We will start from right. Apart from the target vertex of $ABP$ we replace each vertex $v$ of the $ABP$ with $d+1$ many vertices $v^{(0)},v^{(1)}, \dots, v^{(d)}$ going from right to left. Where $v^{(i)}$ computes the $\deg i$ term of the polynomial obtained from the $ABP$ by making $v$ the source vertex and the target vertex is same as before. Here by $v^{(i)}$ we mean that polynomial also. Thus the polynomial obtained by making $s^{(i)}$ the source vertex and target vertex $t$ same as before we get the coefficient of $x^i$ in $\det M$. 
	
	To homogenize let before there was an edge $(u,v)$ with weight $p(x)=a_dx^d+\cdots +a_1x+a_0$. Since $v$ is on right side of $u$, $v$ is replaced with $v^{(0)},v^{(1)}, \dots, v^{(d)}$. Now we first replace $u$ with $u^{(0)},u^{(1)}, \dots, u^{(d)}$. Now obviously we have $u^{(i)}=\sum\limits_{j=0}^i a_j x^j v^{(i-j)}$. So for $0\leq j\leq i$ we join the edges $(u^{(i)},v^{(i-j)})$ with weight $a_jx^j$. We keep on doing this from right to left. 
	
	In the end the source vertex is splited into $d+1$ vertices. Here $s^{(i)}$ computes the coefficient of $x^i$ in $\det M$ multiplied by $x^i$. Now if we replace the variable $x$ in this new homogenized $ABP$ with $1$ then we can say that $s^{(i)}$ computes the coefficient of $x^i$ in $\det M$. So now we reduced that coefficient of $x^i$ is $\det M$ is the value of the $ABP$ whose source vertex is $s^{(i)}$ and target vertex is $t$, same as before. Since $ABP$ is in $GapL$ we have coefficient of $x^i$ in $\det M$ is in $GapL$.
}

	
%%%%%%%%%%%%%%%%%%%%%%%%%%%%%%%%%%%%%%%%%%%%%%%%%%%%%%%%%%%%%%%%%%%%%%%%%%%%%%%%%%%%%%%%%%%%%%%%%%%%%%%%%%%%%%%%%%%%%%%%%%%%%%%%%%%%%%%%
% Problem 3
%%%%%%%%%%%%%%%%%%%%%%%%%%%%%%%%%%%%%%%%%%%%%%%%%%%%%%%%%%%%%%%%%%%%%%%%%%%%%%%%%%%%%%%%%%%%%%%%%%%%%%%%%%%%%%%%%%%%%%%%%%%%%%%%%%%%%%%%

\begin{problem}{%problem statement
	}{p3% problem reference text
	}
	Give a dicirculation of a bidirected graph $G$ using non-vanishing circulation
	
	%Problem		
\end{problem}

\solve{
	%Solution
	Let the bidirected graph is $G=(V,E)$. From $G$ we create a new graph  where for all $(u,v),(v,u)\in E$ we introduce a new vertex $t_{u,v}$ and the edges $\{u,t_{u,v}\}, \{t_{u,v}, v\}$. So this new graph call this $\tdG$. 
	
	Now let $\tdG$ has any non vanishing circulation. So for any edge $\{x,y\}\in E(\tdG)$ we denote the weight of the edge as $w(x,y)$. Now we define the weights in $G$ such that $$w'(u,v)=w(u,t_{u,v})-w(t_{u,v},v)$$ where $w'$ is the weight of the edge $(u,v)\in E$. We claim this is a dicirculation of $G$. To prove let $C$ be any cycle in $G$. Let $C=u_0u_1u_2\cdots u_{2k-1}u_1	$. Then $$w(C)=\sum_{e\in C}w(e)=\sum_{i=0}^{2k-1} w'(u_i, u_{i+1})=\sum_{i=0}^{2k-1} w(u_i, t_{u_i,u_{i+1}})-w(t_{u_i,u_{i+1}},v)\neq 0$$Since this is true for any cycle we obtain a dicirculation.
}

	
	
\end{document}
