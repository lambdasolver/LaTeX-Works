\documentclass[a4paper, 11pt]{article}
\usepackage{comment} % enables the use of multi-line comments (\ifx \fi) 
\usepackage{fullpage} % changes the margin
\usepackage[a4paper, total={7in, 10in}]{geometry}
\usepackage{amsmath,mathtools,mathdots,amssymb}
\usepackage{amssymb,amsthm}  % assumes amsmath package installed
\usepackage{float}
\usepackage{xcolor}
\usepackage{mdframed}
\usepackage[shortlabels]{enumitem}
\usepackage{indentfirst}
\usepackage{hyperref}
\hypersetup{
	colorlinks=true,
	linkcolor=blue,
	filecolor=magenta,      
	urlcolor=blue!70!red,
	pdftitle={Assignment}, %%%%%%%%%%%%%%%%   WRITE ASSIGNMENT PDF NAME  %%%%%%%%%%%%%%%%%%%%
}
\usepackage[most,many,breakable]{tcolorbox}
\usepackage{tikz}
\usetikzlibrary{decorations.pathreplacing,angles,quotes,patterns}
\usetikzlibrary{decorations.shapes}
\usepackage{caption}
%\usepackage{mathpazo}
\usepackage{mathpazo}
\usepackage{libertine}

\definecolor{mytheorembg}{HTML}{F2F2F9}
\definecolor{mytheoremfr}{HTML}{00007B}


\tcbuselibrary{theorems,skins,hooks}
\newtcbtheorem{problem}{Problem}
{%
	enhanced,
	breakable,
	colback = mytheorembg,
	frame hidden,
	boxrule = 0sp,
	borderline west = {2pt}{0pt}{mytheoremfr},
	sharp corners,
	detach title,
	before upper = \tcbtitle\par\smallskip,
	coltitle = mytheoremfr,
	fonttitle = \bfseries\sffamily,
	description font = \mdseries,
	separator sign none,
	segmentation style={solid, mytheoremfr},
}
{p}
\usepackage[ruled,vlined,linesnumbered]{algorithm2e}
\usepackage{physics}
% To give references for any problem use like this
% suppose the problem number is p3 then 2 options either 
% \hyperref[p:p3]{<text you want to use to hyperlink> \ref{p:p3}}
%                  or directly 
%                   \ref{p:p3}

\tikzset{decorate sep/.style 2 args=
	{decorate,decoration={shape backgrounds,shape=circle,shape size=#1,shape sep=#2}}}

%---------------------------------------
% BlackBoard Math Fonts :-
%---------------------------------------

%Captital Letters
\newcommand{\bbA}{\mathbb{A}}	\newcommand{\bbB}{\mathbb{B}}
\newcommand{\bbC}{\mathbb{C}}	\newcommand{\bbD}{\mathbb{D}}
\newcommand{\bbE}{\mathbb{E}}	\newcommand{\bbF}{\mathbb{F}}
\newcommand{\bbG}{\mathbb{G}}	\newcommand{\bbH}{\mathbb{H}}
\newcommand{\bbI}{\mathbb{I}}	\newcommand{\bbJ}{\mathbb{J}}
\newcommand{\bbK}{\mathbb{K}}	\newcommand{\bbL}{\mathbb{L}}
\newcommand{\bbM}{\mathbb{M}}	\newcommand{\bbN}{\mathbb{N}}
\newcommand{\bbO}{\mathbb{O}}	\newcommand{\bbP}{\mathbb{P}}
\newcommand{\bbQ}{\mathbb{Q}}	\newcommand{\bbR}{\mathbb{R}}
\newcommand{\bbS}{\mathbb{S}}	\newcommand{\bbT}{\mathbb{T}}
\newcommand{\bbU}{\mathbb{U}}	\newcommand{\bbV}{\mathbb{V}}
\newcommand{\bbW}{\mathbb{W}}	\newcommand{\bbX}{\mathbb{X}}
\newcommand{\bbY}{\mathbb{Y}}	\newcommand{\bbZ}{\mathbb{Z}}

%---------------------------------------
% MathCal Fonts :-
%---------------------------------------

%Captital Letters
\newcommand{\mcA}{\mathcal{A}}	\newcommand{\mcB}{\mathcal{B}}
\newcommand{\mcC}{\mathcal{C}}	\newcommand{\mcD}{\mathcal{D}}
\newcommand{\mcE}{\mathcal{E}}	\newcommand{\mcF}{\mathcal{F}}
\newcommand{\mcG}{\mathcal{G}}	\newcommand{\mcH}{\mathcal{H}}
\newcommand{\mcI}{\mathcal{I}}	\newcommand{\mcJ}{\mathcal{J}}
\newcommand{\mcK}{\mathcal{K}}	\newcommand{\mcL}{\mathcal{L}}
\newcommand{\mcM}{\mathcal{M}}	\newcommand{\mcN}{\mathcal{N}}
\newcommand{\mcO}{\mathcal{O}}	\newcommand{\mcP}{\mathcal{P}}
\newcommand{\mcQ}{\mathcal{Q}}	\newcommand{\mcR}{\mathcal{R}}
\newcommand{\mcS}{\mathcal{S}}	\newcommand{\mcT}{\mathcal{T}}
\newcommand{\mcU}{\mathcal{U}}	\newcommand{\mcV}{\mathcal{V}}
\newcommand{\mcW}{\mathcal{W}}	\newcommand{\mcX}{\mathcal{X}}
\newcommand{\mcY}{\mathcal{Y}}	\newcommand{\mcZ}{\mathcal{Z}}



%---------------------------------------
% Bold Math Fonts :-
%---------------------------------------

%Captital Letters
\newcommand{\bmA}{\boldsymbol{A}}	\newcommand{\bmB}{\boldsymbol{B}}
\newcommand{\bmC}{\boldsymbol{C}}	\newcommand{\bmD}{\boldsymbol{D}}
\newcommand{\bmE}{\boldsymbol{E}}	\newcommand{\bmF}{\boldsymbol{F}}
\newcommand{\bmG}{\boldsymbol{G}}	\newcommand{\bmH}{\boldsymbol{H}}
\newcommand{\bmI}{\boldsymbol{I}}	\newcommand{\bmJ}{\boldsymbol{J}}
\newcommand{\bmK}{\boldsymbol{K}}	\newcommand{\bmL}{\boldsymbol{L}}
\newcommand{\bmM}{\boldsymbol{M}}	\newcommand{\bmN}{\boldsymbol{N}}
\newcommand{\bmO}{\boldsymbol{O}}	\newcommand{\bmP}{\boldsymbol{P}}
\newcommand{\bmQ}{\boldsymbol{Q}}	\newcommand{\bmR}{\boldsymbol{R}}
\newcommand{\bmS}{\boldsymbol{S}}	\newcommand{\bmT}{\boldsymbol{T}}
\newcommand{\bmU}{\boldsymbol{U}}	\newcommand{\bmV}{\boldsymbol{V}}
\newcommand{\bmW}{\boldsymbol{W}}	\newcommand{\bmX}{\boldsymbol{X}}
\newcommand{\bmY}{\boldsymbol{Y}}	\newcommand{\bmZ}{\boldsymbol{Z}}
%Small Letters
\newcommand{\bma}{\boldsymbol{a}}	\newcommand{\bmb}{\boldsymbol{b}}
\newcommand{\bmc}{\boldsymbol{c}}	\newcommand{\bmd}{\boldsymbol{d}}
\newcommand{\bme}{\boldsymbol{e}}	\newcommand{\bmf}{\boldsymbol{f}}
\newcommand{\bmg}{\boldsymbol{g}}	\newcommand{\bmh}{\boldsymbol{h}}
\newcommand{\bmi}{\boldsymbol{i}}	\newcommand{\bmj}{\boldsymbol{j}}
\newcommand{\bmk}{\boldsymbol{k}}	\newcommand{\bml}{\boldsymbol{l}}
\newcommand{\bmm}{\boldsymbol{m}}	\newcommand{\bmn}{\boldsymbol{n}}
\newcommand{\bmo}{\boldsymbol{o}}	\newcommand{\bmp}{\boldsymbol{p}}
\newcommand{\bmq}{\boldsymbol{q}}	\newcommand{\bmr}{\boldsymbol{r}}
\newcommand{\bms}{\boldsymbol{s}}	\newcommand{\bmt}{\boldsymbol{t}}
\newcommand{\bmu}{\boldsymbol{u}}	\newcommand{\bmv}{\boldsymbol{v}}
\newcommand{\bmw}{\boldsymbol{w}}	\newcommand{\bmx}{\boldsymbol{x}}
\newcommand{\bmy}{\boldsymbol{y}}	\newcommand{\bmz}{\boldsymbol{z}}

%---------------------------------------
% Scr Math Fonts :-
%---------------------------------------

\newcommand{\sA}{{\mathscr{A}}}   \newcommand{\sB}{{\mathscr{B}}}
\newcommand{\sC}{{\mathscr{C}}}   \newcommand{\sD}{{\mathscr{D}}}
\newcommand{\sE}{{\mathscr{E}}}   \newcommand{\sF}{{\mathscr{F}}}
\newcommand{\sG}{{\mathscr{G}}}   \newcommand{\sH}{{\mathscr{H}}}
\newcommand{\sI}{{\mathscr{I}}}   \newcommand{\sJ}{{\mathscr{J}}}
\newcommand{\sK}{{\mathscr{K}}}   \newcommand{\sL}{{\mathscr{L}}}
\newcommand{\sM}{{\mathscr{M}}}   \newcommand{\sN}{{\mathscr{N}}}
\newcommand{\sO}{{\mathscr{O}}}   \newcommand{\sP}{{\mathscr{P}}}
\newcommand{\sQ}{{\mathscr{Q}}}   \newcommand{\sR}{{\mathscr{R}}}
\newcommand{\sS}{{\mathscr{S}}}   \newcommand{\sT}{{\mathscr{T}}}
\newcommand{\sU}{{\mathscr{U}}}   \newcommand{\sV}{{\mathscr{V}}}
\newcommand{\sW}{{\mathscr{W}}}   \newcommand{\sX}{{\mathscr{X}}}
\newcommand{\sY}{{\mathscr{Y}}}   \newcommand{\sZ}{{\mathscr{Z}}}


%---------------------------------------
% Math Fraktur Font
%---------------------------------------

%Captital Letters
\newcommand{\mfA}{\mathfrak{A}}	\newcommand{\mfB}{\mathfrak{B}}
\newcommand{\mfC}{\mathfrak{C}}	\newcommand{\mfD}{\mathfrak{D}}
\newcommand{\mfE}{\mathfrak{E}}	\newcommand{\mfF}{\mathfrak{F}}
\newcommand{\mfG}{\mathfrak{G}}	\newcommand{\mfH}{\mathfrak{H}}
\newcommand{\mfI}{\mathfrak{I}}	\newcommand{\mfJ}{\mathfrak{J}}
\newcommand{\mfK}{\mathfrak{K}}	\newcommand{\mfL}{\mathfrak{L}}
\newcommand{\mfM}{\mathfrak{M}}	\newcommand{\mfN}{\mathfrak{N}}
\newcommand{\mfO}{\mathfrak{O}}	\newcommand{\mfP}{\mathfrak{P}}
\newcommand{\mfQ}{\mathfrak{Q}}	\newcommand{\mfR}{\mathfrak{R}}
\newcommand{\mfS}{\mathfrak{S}}	\newcommand{\mfT}{\mathfrak{T}}
\newcommand{\mfU}{\mathfrak{U}}	\newcommand{\mfV}{\mathfrak{V}}
\newcommand{\mfW}{\mathfrak{W}}	\newcommand{\mfX}{\mathfrak{X}}
\newcommand{\mfY}{\mathfrak{Y}}	\newcommand{\mfZ}{\mathfrak{Z}}
%Small Letters
\newcommand{\mfa}{\mathfrak{a}}	\newcommand{\mfb}{\mathfrak{b}}
\newcommand{\mfc}{\mathfrak{c}}	\newcommand{\mfd}{\mathfrak{d}}
\newcommand{\mfe}{\mathfrak{e}}	\newcommand{\mff}{\mathfrak{f}}
\newcommand{\mfg}{\mathfrak{g}}	\newcommand{\mfh}{\mathfrak{h}}
\newcommand{\mfi}{\mathfrak{i}}	\newcommand{\mfj}{\mathfrak{j}}
\newcommand{\mfk}{\mathfrak{k}}	\newcommand{\mfl}{\mathfrak{l}}
\newcommand{\mfm}{\mathfrak{m}}	\newcommand{\mfn}{\mathfrak{n}}
\newcommand{\mfo}{\mathfrak{o}}	\newcommand{\mfp}{\mathfrak{p}}
\newcommand{\mfq}{\mathfrak{q}}	\newcommand{\mfr}{\mathfrak{r}}
\newcommand{\mfs}{\mathfrak{s}}	\newcommand{\mft}{\mathfrak{t}}
\newcommand{\mfu}{\mathfrak{u}}	\newcommand{\mfv}{\mathfrak{v}}
\newcommand{\mfw}{\mathfrak{w}}	\newcommand{\mfx}{\mathfrak{x}}
\newcommand{\mfy}{\mathfrak{y}}	\newcommand{\mfz}{\mathfrak{z}}

%---------------------------------------
% Bar
%---------------------------------------

%Captital Letters
\newcommand{\barA}{\overline{A}}	\newcommand{\barB}{\overline{B}}
\newcommand{\barC}{\overline{C}}	\newcommand{\barD}{\overline{D}}
\newcommand{\barE}{\overline{E}}	\newcommand{\barF}{\overline{F}}
\newcommand{\barG}{\overline{G}}	\newcommand{\barH}{\overline{H}}
\newcommand{\barI}{\overline{I}}	\newcommand{\barJ}{\overline{J}}
\newcommand{\barK}{\overline{K}}	\newcommand{\barL}{\overline{L}}
\newcommand{\barM}{\overline{M}}	\newcommand{\barN}{\overline{N}}
\newcommand{\barO}{\overline{O}}	\newcommand{\barP}{\overline{P}}
\newcommand{\barQ}{\overline{Q}}	\newcommand{\barR}{\overline{R}}
\newcommand{\barS}{\overline{S}}	\newcommand{\barT}{\overline{T}}
\newcommand{\barU}{\overline{U}}	\newcommand{\barV}{\overline{V}}
\newcommand{\barW}{\overline{W}}	\newcommand{\barX}{\overline{X}}
\newcommand{\barY}{\overline{Y}}	\newcommand{\barZ}{\overline{Z}}
%Small Letters
\newcommand{\bara}{\overline{a}}	\newcommand{\barb}{\overline{b}}
\newcommand{\barc}{\overline{c}}	\newcommand{\bard}{\overline{d}}
\newcommand{\bare}{\overline{e}}	\newcommand{\barf}{\overline{f}}
\newcommand{\barg}{\overline{g}}	\newcommand{\barh}{\overline{h}}
\newcommand{\bari}{\overline{i}}	\newcommand{\barj}{\overline{j}}
\newcommand{\bark}{\overline{k}}	\newcommand{\barl}{\overline{l}}
\newcommand{\barm}{\overline{m}}	\newcommand{\barn}{\overline{n}}
\newcommand{\baro}{\overline{o}}	\newcommand{\barp}{\overline{p}}
\newcommand{\barq}{\overline{q}}	\newcommand{\barr}{\overline{r}}
\newcommand{\bars}{\overline{s}}	\newcommand{\bart}{\overline{t}}
\newcommand{\baru}{\overline{u}}	\newcommand{\barv}{\overline{v}}
\newcommand{\barw}{\overline{w}}	\newcommand{\barx}{\overline{x}}
\newcommand{\bary}{\overline{y}}	\newcommand{\barz}{\overline{z}}

%---------------------------------------
% Greek Letters:-
%---------------------------------------
\newcommand{\eps}{\epsilon}
\newcommand{\veps}{\varepsilon}
\newcommand{\lm}{\lambda}
\newcommand{\Lm}{\Lambda}
\newcommand{\gm}{\gamma}
\newcommand{\Gm}{\Gamma}
\newcommand{\vph}{\varphi}
\newcommand{\ph}{\phi}

\newcommand{\Qed}{\begin{flushright}\qed\end{flushright}}
\newcommand{\parinn}{\setlength{\parindent}{1cm}}
\newcommand{\parinf}{\setlength{\parindent}{0cm}}
\newcommand{\del}[2]{\frac{\partial #1}{\partial #2}}
\newcommand{\Del}[3]{\frac{\partial^{#1} #2}{\partial^{#1} #3}}
\newcommand{\deld}[2]{\dfrac{\partial #1}{\partial #2}}
\newcommand{\Deld}[3]{\dfrac{\partial^{#1} #2}{\partial^{#1} #3}}
\newcommand{\uin}{\mathbin{\rotatebox[origin=c]{90}{$\in$}}}
\newcommand{\usubset}{\mathbin{\rotatebox[origin=c]{90}{$\subset$}}}
\newcommand{\lt}{\left}
\newcommand{\rt}{\right}
\newcommand{\exs}{\exists}
\newcommand{\st}{\strut}
\newcommand{\dps}[1]{\displaystyle{#1}}
\newcommand{\la}{\langle}
\newcommand{\ra}{\rangle}
\newenvironment{solution}
{\textit{\textbf{Solution:}} 
}
{ 
	\hfill $\blacksquare$
	
	\vspace{1cm}
}
\newcommand{\sol}[1]{\begin{solution}#1\end{solution}}
\newcommand{\solve}[1]{\setlength{\parindent}{0cm}\textbf{\textit{Solution: }}\setlength{\parindent}{1cm}#1 \Qed}
\newcommand{\mat}[1]{\left[\begin{matrix}#1\end{matrix}\right]}
\newcommand\numberthis{\addtocounter{equation}{1}\tag{\theequation}}
\newcommand{\handout}[3]{
	\noindent
	\begin{center}
		\framebox{
			\vbox{
				\hbox to 6.5in { {\bf Complexity Theory I } \hfill Jan -- May, 2023 }
				\vspace{4mm}
				\hbox to 6.5in { {\Large \hfill #1  \hfill} }
				\vspace{2mm}
				\hbox to 6.5in { {\em #2 \hfill #3} }
			}
		}
	\end{center}
	\vspace*{4mm}
}

\newcommand{\lecture}[3]{\handout{Lecture #1}{Lecturer: #2}{Scribe:	#3}}
	

\newcommand{\ov}[1]{\overline{#1}}
\newcommand{\thmref}[1]{\hyperref[#1]{Theorem \ref{#1}}}

\setlength{\parindent}{0pt}

%%%%%%%%%%%%%%%%%%%%%%%%%%%%%%%%%%%%%%%%%%%%%%%%%%%%%%%%%%%%%%%%%%%%%%%%%%%%%%%%%%%%%%%%%%%%%%%%%%%%%%%%%%%%%%%%%%%%%%%%%%%%%%%%%%%%%%%%

\begin{document}
	
%%%%%%%%%%%%%%%%%%%%%%%%%%%%%%%%%%%%%%%%%%%%%%%%%%%%%%%%%%%%%%%%%%%%%%%%%%%%%%%%%%%%%%%%%%%%%%%%%%%%%%%%%%%%%%%%%%%%%%%%%%%%%%%%%%%%%%%%
	
\textsf{\noindent \large\textbf{Soham Chatterjee} \hfill \textbf{Assignment - 1}\\
	Email: \href{sohamc@cmi.ac.in}{sohamc@cmi.ac.in} \hfill Roll: BMC202175\\
	\normalsize Course: Quantum Algorithmic Thinking \hfill Date: October 20, 2023}
	
%%%%%%%%%%%%%%%%%%%%%%%%%%%%%%%%%%%%%%%%%%%%%%%%%%%%%%%%%%%%%%%%%%%%%%%%%%%%%%%%%%%%%%%%%%%%%%%%%%%%%%%%%%%%%%%%%%%%%%%%%%%%%%%%%%%%%%%%
% Problem 1
%%%%%%%%%%%%%%%%%%%%%%%%%%%%%%%%%%%%%%%%%%%%%%%%%%%%%%%%%%%%%%%%%%%%%%%%%%%%%%%%%%%%%%%%%%%%%%%%%%%%%%%%%%%%%%%%%%%%%%%%%%%%%%%%%%%%%%%%
	
\begin{problem}{%problem statement
	}{p1% problem reference text
	}
Find the eigenvectors, eigenvalues, and diagonal representations of the Pauli matrices.		
%Problem		
\end{problem}
	
\solve{
	%Solution
	Pauli matrices are $$I=\mat{1&0\\ 0&1}\quad \sg_x=\mat{0&1\\ 1&0}\quad \sg_y=\mat{0& i \\ -i& 0}\quad \sg_z=\mat{1&0\\ 0&-1}$$ For $I$ for all vectors $v$ $Iv=v$. S0 every vector is an eigenvector and its eigenvalue is 1. Since $I$ is already in its diagonal representation $I$'s diagonal representation is $I$ itself.
	
	Since $\sg_x\mat{1\\0}=\mat{0\\1}$ and $\sg_x\mat{0\\1}=\mat{1\\ 0}$ we have 
	$$\sg_x\lt(\mat{1\\ 0}+\mat{0\\ 1}\rt)=\mat{0\\ 1}+\mat{1\\ 0}\quad \sg_x\lt(\mat{1\\ 0} -\mat{0\\ 1}\rt)=\mat{0\\ 1}-\mat{1\\ 0}=-\lt(\mat{1\\ 0}-\mat{0\\ 1}\rt)$$So the for the eignevalue $1$ the corresponding eignevector is $\mat{1\\ 0}+\mat{0\\ 1}$ and for the eigenvalue $-1$ the corresponding eigenvalue is $\mat{1\\ 0}-\mat{0\\ 1}$.
	
	Since $\sg_y\mat{1\\ 0}=\mat{0\\ -i}$ and $\sg_y\mat{0\\ 1}=\mat{i\\ 0}$ we have 
	$$\sg_y\lt(\mat{1\\ 0}+i\mat{0\\ 1}\rt)=\mat{0\\ -i}+i\mat{i\\ 0} = -1\lt(i\mat{0\\ 1}+\mat{1\\ 0}\rt)\quad \sg_y\lt(\mat{1\\ 0} -i\mat{0\\ 1}\rt)=\mat{0\\ -i}-i\mat{i\\ 0}=-i\mat{0\\ 1}+\mat{1\\ 0}$$So the for the eignevalue $1$ the corresponding eignevector is $\mat{1\\ 0} -i\mat{0\\ 1}$ and for the eigenvalue $-1$ the corresponding eigenvalue is $\mat{1\\ 0}+i\mat{0\\ 1}$.
	
	Since $\sg_z\mat{1\\ 0}=\mat{1\\ 0}$ and $\sg_y\mat{0\\ 1}=-\mat{0\\ 1}$. So the for the eignevalue $1$ the corresponding eignevector is $\mat{1\\ 0}$ and for the eigenvalue $-1$ the corresponding eigenvalue is $\mat{0\\ 1}$.
	
	Now $\sg_x,\sg_y,\sg_z$ has eigenvalues 1 and -1. So if we write in their corresponding eigenbasis then we will obtain the same diagonalized matrices where all the eigenvalues are in the diagonal positions i.e. $\mat{1 &0\\ 0& -1}$
}
	
%%%%%%%%%%%%%%%%%%%%%%%%%%%%%%%%%%%%%%%%%%%%%%%%%%%%%%%%%%%%%%%%%%%%%%%%%%%%%%%%%%%%%%%%%%%%%%%%%%%%%%%%%%%%%%%%%%%%%%%%%%%%%%%%%%%%%%%%
% Problem 2
%%%%%%%%%%%%%%%%%%%%%%%%%%%%%%%%%%%%%%%%%%%%%%%%%%%%%%%%%%%%%%%%%%%%%%%%%%%%%%%%%%%%%%%%%%%%%%%%%%%%%%%%%%%%%%%%%%%%%%%%%%%%%%%%%%%%%%%%

\begin{problem}{%problem statement
	}{p2% problem reference text
	}
	Show that a normal matrix is Hermitian if and only if it has real eigenvalues. Show that a positive operator is necessarily Hermitian.
	%Problem		
\end{problem}

\solve{
	%Solution
	Let $A$ is normal and it is hermitian. Then $A=A^{\dagger}$. Let $v$ be an eigenvector of $A$ with eigenvalue $\lm$.  Then $v^{\dagger}Av=v^{\dagger}\lm v=\lm |v|^2$. Also $v^{\dagger}Av=v^{\dagger}A^{\dagger}v=(Av)^{\dagger} v=\lm^{\dagger}v^{\dagger}v=\lm^{\dagger} |v|^2$. So we have $\lm=\lm^{\dagger}$. Which implies $\lm$ is real. Hence all eigenvalues of $A$ are real.
	
	For the opposite direction we need some lemmas.
	\parinf
	
	\textbf{\textit{Lemma 1:}} The product of two unitary matrices is unitary
	
	\textbf{\textit{Proof:}} Let $U,V$ are two unitary matrices then $(UV)^{\dagger}=V^{\dagger} U^{\dagger}$. Now $(UV)(UV)^{\dagger}=U(VV^{\dagger}U^{\dagger})=UIU^{\dagger}=I$. 
	
	\textbf{\textit{Lemma 2:}} If $A$ is any square complex matrix then there is an upper triangular complex matrix $T$ and a unitary matrix $U$ so that $A=UTU^{\dagger}$
	
	\textbf{\textit{Proof:}} Let $A$ is a $n\times n$ matrix. Let $v_1$ be a eigenvector of $A$ with the corresponding eigenvalue $\lm_1$. We can take $x_1$ to be of unit length. Now by Gram-Schmidt process  we can extend $x_1$ to an orthonormal basis $\{x_1,v_2,\dots,v_n\}$; Let $S_0=\mat{x_1 & v_2 & \cdots & v_n}$  then $S_0$ is unitary and $$S_0^{\dagger}AS_0=\mat{\lm_1 & * \\ 0 & A_1}$$ where $A_1$ is an $(n-1)\times (n-1)$ matrix.  Again suppose $x_2$ is an eigenvector of $A_1$ and the corresponding eigenvalue is $\lm_2$. Then again for $A_1$ we extend $x_2$ to an orthonormal basis $\{x_2,\tdv_2,\dots, \tdv_{n-1}\}$ and take $\hat{S}_1=\mat{x_2,\tdv_2,\cdots, \tdv_{n-1}}$ then $S_1$ is also unitary and we have $\hat{S}_1^{\dagger}A_1\hat{S}_1=\mat{\lm_2 & * \\ 0 & A_2}$ where $A_2$ is a $(n-2)\times (n-2)$ matrix.  So we take $S_1=S_0\mat{1 & 0\\ 0 & \hat{S}_1}$. Then $$S_1^{\dagger} AS_1=\mat{\lm_1 & * & *\\ 0 & \lm_2 & *\\ 0 & 0& A_2}$$We continue like this  letting $S_k=S_{k-1}\mat{I_k & 0 \\ 0 & \hat{S}_{k}}$ thus at the end we obtain $U\coloneqq S_n$ such that $U^{\dagger}AU=T$ which is an upper triangular matrix. Hence we have $A=UTU^{\dagger}$
	
	\textbf{\textit{Lemma 3:}} A matrix $A$ is diagonalizable with a unitary matrix if and only if $A$ is normal
	
	\textbf{\textit{Proof:}} Let $A$ is normal. Then by Lemma 2 there  is a unitary matrix $U$ and a upper traingular matrix $T$ such that $A=UTU^{\dagger}$. Then \begin{multline*}
		TT^{\dagger}=U^{\dagger}AU(U^{\dagger}AU)^{\dagger}=U^{\dagger}AUU^{\dagger} A^{\dagger} U	=U^{\dagger}A A^{\dagger} U\\
	=U^{\dagger} A^{\dagger}A U=U^{\dagger} A^{\dagger}UU^{\dagger}A U=(U^{\dagger}AU)^{\dagger}U^{\dagger}AU=T^{\dagger}T
	\end{multline*}Now let $T+(t_{i,j})_{1\leq i,j\leq n}$. Then the first diagonal entry of $TT^{\dagger}$ is $$\sum_{i=1}^n  t_{1,i} \overline{t_{1,i}}=\sum_{i=1}^n |t_{1,i}|^2$$ Now the first diagonal entry of $T^{\dagger}T$ is $t_{1,1}\ov{t_{1,1}}=|t_{1,1}|^2$. These two are equal. Hence for all $2\leq i\leq n$ we have $t_{1,i}=0$. Similarly comparing the second diagonal entry of $TT^{\dagger} $ and $T^{\dagger}T$ we have that all the nondiagonal entries of second row of $T$ is 0.  Continuing like this we have that $T$ is diagonal. 
	
	Now suppose that $A$ is any matrix such that there exists an unitary matrix $U$ such that  $U^{\dagger}AU=D$ where $D$ is diagonal. Then \begin{multline*}
		AA^{\dagger}=UDU^{\dagger}(UDU^{\dagger})^{\dagger}=UDU^{\dagger}UD^{\dagger}U^{\dagger}=UDD^{\dagger}U^{\dagger}\\
		=UD^{\dagger}DU^{\dagger}=UD^{\dagger}U^{\dagger}UDU^{\dagger}=(UDU^{\dagger})^{\dagger}UDU^{\dagger}=A^{\dagger}A
	\end{multline*}So $A$ is normal.
	
	
	\parinn
	
	Now coming back to the original question we have that the eigenvalues of $A$ are real. $A$ is normal. Then  there exists an unitary matrix $U$ such that  $U^{\dagger}AU=D$ where $D$ is diagonal. Since all eigenvalues of $A$ are real $D^{\dagger}=D$. Then we have $$A^{\dagger}=(U^{\dagger}DU)^{\dagger}=U^{\dagger}D^{\dagger}U=U^{\dagger}DU=A$$So $A$ is hermitian
	
	\vspace{5mm}
	
	Now suppose $A$ is positive operator. Then for all $v\in V$ we have $$v^{\dagger}Av\geq 0\implies v^{\dagger}Av= (v^{\dagger}Av)^{\dagger}=v^{\dagger}A^{\dagger}v\geq 0\implies v^{\dagger}(A-A^{\dagger})v=0$$Now also we have \begin{multline*}
		(A-A^{\dagger})(A-A^{\dagger})^{\dagger}=(A-A^{\dagger})(A^{\dagger}-A)=AA^{\dagger}-A^{\dagger}A^{\dagger}-AA+A^{\dagger}A\\
		=(A^{\dagger}-A)(A-A^{\dagger})=(A-A^{\dagger})^{\dagger}(A-A^{\dagger})
	\end{multline*}So $A-A^{\dagger}$ is a normal operator. Hence by Lemma 3 there exists an unitary matrix $U$ such that $U^{\dagger}(A-A^{\dagger})U=D$ where $D$ is a diagonal matrix. Now for standard basis for any $e_i$ $$e_i^{\dagger}De_i=e^{\dagger}U^{\dagger}(A-A^{\dagger})Ue_i=(Ue_i)^{\dagger}(A-A^{\dagger})(Ue_i)= 0$$ Now $e_i^{\dagger}De_i$ is the $i$-th diagonal element of $D$ which we got is 0. Since this is true for all $i\in [n]$ we have $D$ is a null matrix. So $$U^{\dagger}(A-A^{\dagger})U=0\iff A-A^{\dagger}=U0U^{\dagger}=0\iff A=A^{\dagger}$$Hence $A$ is hermitian.
}


%%%%%%%%%%%%%%%%%%%%%%%%%%%%%%%%%%%%%%%%%%%%%%%%%%%%%%%%%%%%%%%%%%%%%%%%%%%%%%%%%%%%%%%%%%%%%%%%%%%%%%%%%%%%%%%%%%%%%%%%%%%%%%%%%%%%%%%%
% Problem 3
%%%%%%%%%%%%%%%%%%%%%%%%%%%%%%%%%%%%%%%%%%%%%%%%%%%%%%%%%%%%%%%%%%%%%%%%%%%%%%%%%%%%%%%%%%%%%%%%%%%%%%%%%%%%%%%%%%%%%%%%%%%%%%%%%%%%%%%%

\begin{problem}{%problem statement
	}{p1% problem reference text
	}
	Suppose that $A$ and $B$ are Hermitian operators. Then show that the commutator $[A, B] = 0$ if and only if there exists an orthonormal basis such that both $A$ and $B$ are diagonal with respect to that basis. 
	%Problem		
\end{problem}

\solve{
	%Solution
	If there exists an orthonormal basis such that both $A$ and $B$ are diagonal with respect to that basis then let we have $P^{\dagger}AP=D_A$ and $P^{\dagger}P-D_B$. Then $$AB-BA=PD_AP^{\dagger}PD_BP^{\dagger}-PD_BP^{\dagger}PD_AP^{\dagger}=PD_AD_BP^{\dagger}-PD_BD_AP^{\dagger}=P(D_AD_B-S_BD_A)P^{\dagger}=0$$The last equality comes because $D_A$ and $D_B$ are diagonal matrices so $D_AD_B=D_BD_A$. 
	
	For the opposite direction suppose $v$ be an eigenvector with corresponding eigenvector $\lm$ of $A$ then $Av=\lm v$. Now $$A(Bv)=BAv=B\lm v=\lm Bv$$Hence for any eigenvector $v$ of $A$ $Bv$ is also an eigenvector and if $Bv$ is zero then still it is an eigenvector of $A$ for same eigenvalue. 
	
	Let $\lm_1,\dots,\lm_k$ be the eigenvalues of $A$. Then the corresponding eigenspaces of $A$ are $V_{\lm_i}$ for $i\in [k]$. Then we have $B(V_{\lm_i})\subseteq V_{\lm_i}$ for all $i\in [k]$. Now let $\beta$ be an eigenvalue of $B$ with corresponding eigenvector is $y$. Then for any $i\in [k]$ we can think $y=y_1+y_2$ where $y_1\in V_{\lm_i}$ and and $y_2\in \bigoplus_{j\neq i}V_{\lm_j}$. Then $By=\beta y=\beta y_1+\beta y_2$. also we have $By=By_2+By_2$. Since $B(V_{\lm_i})\subseteq V_{\lm_i}$ and $B\lt( \bigoplus_{j\neq i}V_{\lm_j} \rt)\subseteq \bigoplus_{j\neq i}V_{\lm_j}$ we can say $By_1=\beta y_1$ and $By_2=\beta y_2$. Now if the $V_{\beta}$ is the corresponding eigenspace fo the eigenvalue $\beta$ then $$V_{\beta}=\lt[ V_{\beta}\cap V_{\lm_i}\rt]\oplus \lt[V_{\beta}\cap \bigoplus_{j\neq i}V_{\lm_j} \rt]=\bigoplus_{i=1}^kV_{\lm_i}\cap V_{\beta}$$Now if $\beta_1,\dots, \beta_l$ are the eigenvalues of $B$ then we have $$\bigoplus_{i=1}^lV_{\beta_i}=\bigoplus_{i=1}^l\lt(\bigoplus_{j=1}^k V_{\lm_j}\cap V_{\beta_i}\rt)=\bigoplus_{\substack{1\leq i\leq l\\ 1\leq j\leq k}}V_{\beta_i}\cap V_{\lm_j}$$Let us denote $V_{i,j}=V_{\beta_i}\cap V_{\lm_j}$ then for each $V_{i,j}$ we take an orthogonal basis for all $i,j$. Then taking union of all of them we have an orthogoanl basis for both $A$ and $B$ such that both $A$ and $B$ are diagonal. Now for each vector in the basis after normalizing we get an orthonormal basis  such that both $A$ and $B$ are diagonal with respect to that basis. 
}


%%%%%%%%%%%%%%%%%%%%%%%%%%%%%%%%%%%%%%%%%%%%%%%%%%%%%%%%%%%%%%%%%%%%%%%%%%%%%%%%%%%%%%%%%%%%%%%%%%%%%%%%%%%%%%%%%%%%%%%%%%%%%%%%%%%%%%%%
% Problem 4
%%%%%%%%%%%%%%%%%%%%%%%%%%%%%%%%%%%%%%%%%%%%%%%%%%%%%%%%%%%%%%%%%%%%%%%%%%%%%%%%%%%%%%%%%%%%%%%%%%%%%%%%%%%%%%%%%%%%%%%%%%%%%%%%%%%%%%%%

\begin{problem}{%problem statement
	}{p1% problem reference text
	}
	Prove that a state $\ket{\psi}$ of a composite system $AB$ is a product state if and only if it has Schmidt number 1. Prove that $\ket{\psi}$ is a product state if and only if the reduced density matrices $\rho_A$ and $\rho_B$ are pure states. 
	%Problem		
\end{problem}

\solve{
	%Solution
	\begin{itemize}
		\item Let the $\ket{\psi}$ is a product state. Then $\exs\ \ket{\psi_1}\in A, \ket{\psi_2}\in B$ such that $\ket{\psi}=\ket{\psi_1}\ket{\psi_2}$. Now by Schmidt Decomposition there exists an orthonormal basis $\{\ket{i_A}\}$ for system $A$ and orthonormal basis $\{\ket{i_B}\}$ for system $B$ such that $$\ket{\psi}=\sum_{i=1}^n\lm_i\ket{i_A}\ket{i_B}$$ where $\lm_i\in \bbR$ such that $\sum_{i=1}^n\lm_i^2=1$. We have there exists at least one $\lm_i\neq 0$. WLOG $\lm_1\neq 0$  Now we also have $$\ket{\psi_1}=\sum_{i=1}^n \lm_{i,A}\ket{i_A}\qquad \ket{\psi_2}=\sum_{i=1}^n\lm_{i,B}\ket{i_B}$$then we have $$\sum_{i=1}^n\lm_i\ket{i_A}\ket{i_B}=\ket{\psi}=\lt(\sum_{i=1}^n\lm_{i,A}\ket{i_A}\rt)\lt(\sum_{i=1}^n\lm_{i,B}\ket{i_B}\rt)=\sum_{1\leq i,j\leq n}\lm_{i,A}\lm_{j,B}\ket{i_A}\ket{j_B}$$Comparing the coefficients we have $\lm_i=\lm_{i,A}\lm_{i,B}$ and for all $\lm_{i,A}\lm_{j,B}=0$ where $i\neq j$. Since $\lm_1\neq 0$ we have $\lm_{1,A},\lm_{1,B}\neq 0$. Since for all $j\neq 1$, $\lm_{1,A}\lm_{j,B}=0$ we have $\lm_{j,B}=0$ for all $2\leq j\leq n$. Similarly since for all $i\neq 1$, $\lm_{i,A}\lm_{1,B}=0$ we have $\lm_{i,A}=0$ for all $2\leq i\leq n$. So we have $\lm_i=0$ for all $2\leq i\leq n$. So $\ket{\psi}=\lm_1\ket{i_A}\ket{i_B}$. Hence $\ket{\psi}$ has Schmidt Number 1.
		\parinn
		
		For the opposite direction $\ket{\psi}$ has Schmidt Number 1. So $\ket{\psi}=\ket{i_A}\ket{i_B}$ Here are $\ket{i_A}$ is a state of system $A$ and $\ket{i_B}$ is a state of system $B$. Hence $\ket{\psi}$ is already in a product state. Hence $\ket{\psi}$ is a product state of the composite system $AB$.
		
		\item 
	\end{itemize}
}


%%%%%%%%%%%%%%%%%%%%%%%%%%%%%%%%%%%%%%%%%%%%%%%%%%%%%%%%%%%%%%%%%%%%%%%%%%%%%%%%%%%%%%%%%%%%%%%%%%%%%%%%%%%%%%%%%%%%%%%%%%%%%%%%%%%%%%%%
% Problem 5
%%%%%%%%%%%%%%%%%%%%%%%%%%%%%%%%%%%%%%%%%%%%%%%%%%%%%%%%%%%%%%%%%%%%%%%%%%%%%%%%%%%%%%%%%%%%%%%%%%%%%%%%%%%%%%%%%%%%%%%%%%%%%%%%%%%%%%%%

\begin{problem}{%problem statement
	}{p1% problem reference text
	}
	Write a self-contained proof that single qubit gates and $CNOT$ gates are universal.
	%Problem		
\end{problem}

\solve{
	%Solution
}


%%%%%%%%%%%%%%%%%%%%%%%%%%%%%%%%%%%%%%%%%%%%%%%%%%%%%%%%%%%%%%%%%%%%%%%%%%%%%%%%%%%%%%%%%%%%%%%%%%%%%%%%%%%%%%%%%%%%%%%%%%%%%%%%%%%%%%%%
% Problem 6
%%%%%%%%%%%%%%%%%%%%%%%%%%%%%%%%%%%%%%%%%%%%%%%%%%%%%%%%%%%%%%%%%%%%%%%%%%%%%%%%%%%%%%%%%%%%%%%%%%%%%%%%%%%%%%%%%%%%%%%%%%%%%%%%%%%%%%%%

\begin{problem}{%problem statement
	}{p1% problem reference text
	}
	Find the eigenvectors, eigenvalues, and diagonal representations of the Pauli matrices.		
	%Problem		
\end{problem}

\solve{
	%Solution
}

	
\end{document}
