%% This is a template for Assignments 
%% My name is Soham Chatterjee
%% I have created this theme.
%% Thanks for using it.


\documentclass[a4paper, 11pt]{article}
\usepackage{comment} % enables the use of multi-line comments (\ifx \fi) 
\usepackage{fullpage} % changes the margin
\usepackage[a4paper, total={7in, 10in}]{geometry}
\usepackage{amsmath,mathtools}
\usepackage{amssymb,amsthm}  % assumes amsmath package installed
\usepackage{float}
\usepackage{graphicx}
\graphicspath{{./images/}}
\usepackage{xcolor}
\usepackage{mdframed}
\usepackage[shortlabels]{enumitem}
\usepackage{indentfirst}
\usepackage{hyperref}
\hypersetup{
	colorlinks=true,
	linkcolor=blue,
	filecolor=magenta,      
	urlcolor=blue!70!red,
	pdftitle={Assignment}, %%%%%%%%%%%%%%%%   WRITE ASSIGNMENT PDF NAME  %%%%%%%%%%%%%%%%%%%%
}
\usepackage[most,many,breakable]{tcolorbox}
\usepackage{mathpazo}


\definecolor{mytheorembg}{HTML}{F2F2F9}
\definecolor{mytheoremfr}{HTML}{00007B}


\tcbuselibrary{theorems,skins,hooks}
\newtcbtheorem{problem}{Problem}
{%
	enhanced,
	breakable,
	colback = mytheorembg,
	frame hidden,
	boxrule = 0sp,
	borderline west = {2pt}{0pt}{mytheoremfr},
	sharp corners,
	detach title,
	before upper = \tcbtitle\par\smallskip,
	coltitle = mytheoremfr,
	fonttitle = \bfseries\sffamily,
	description font = \mdseries,
	separator sign none,
	segmentation style={solid, mytheoremfr},
}
{p}

% To give references for any problem use like this
% suppose the problem number is p3 then 2 options either 
% \hyperref[p:p3]{<text you want to use to hyperlink> \ref{p:p3}}
%                  or directly 
%                   \ref{p:p3}



%---------------------------------------
% BlackBoard Math Fonts :-
%---------------------------------------

%Captital Letters
\newcommand{\bbA}{\mathbb{A}}	\newcommand{\bbB}{\mathbb{B}}
\newcommand{\bbC}{\mathbb{C}}	\newcommand{\bbD}{\mathbb{D}}
\newcommand{\bbE}{\mathbb{E}}	\newcommand{\bbF}{\mathbb{F}}
\newcommand{\bbG}{\mathbb{G}}	\newcommand{\bbH}{\mathbb{H}}
\newcommand{\bbI}{\mathbb{I}}	\newcommand{\bbJ}{\mathbb{J}}
\newcommand{\bbK}{\mathbb{K}}	\newcommand{\bbL}{\mathbb{L}}
\newcommand{\bbM}{\mathbb{M}}	\newcommand{\bbN}{\mathbb{N}}
\newcommand{\bbO}{\mathbb{O}}	\newcommand{\bbP}{\mathbb{P}}
\newcommand{\bbQ}{\mathbb{Q}}	\newcommand{\bbR}{\mathbb{R}}
\newcommand{\bbS}{\mathbb{S}}	\newcommand{\bbT}{\mathbb{T}}
\newcommand{\bbU}{\mathbb{U}}	\newcommand{\bbV}{\mathbb{V}}
\newcommand{\bbW}{\mathbb{W}}	\newcommand{\bbX}{\mathbb{X}}
\newcommand{\bbY}{\mathbb{Y}}	\newcommand{\bbZ}{\mathbb{Z}}

%---------------------------------------
% MathCal Fonts :-
%---------------------------------------

%Captital Letters
\newcommand{\mcA}{\mathcal{A}}	\newcommand{\mcB}{\mathcal{B}}
\newcommand{\mcC}{\mathcal{C}}	\newcommand{\mcD}{\mathcal{D}}
\newcommand{\mcE}{\mathcal{E}}	\newcommand{\mcF}{\mathcal{F}}
\newcommand{\mcG}{\mathcal{G}}	\newcommand{\mcH}{\mathcal{H}}
\newcommand{\mcI}{\mathcal{I}}	\newcommand{\mcJ}{\mathcal{J}}
\newcommand{\mcK}{\mathcal{K}}	\newcommand{\mcL}{\mathcal{L}}
\newcommand{\mcM}{\mathcal{M}}	\newcommand{\mcN}{\mathcal{N}}
\newcommand{\mcO}{\mathcal{O}}	\newcommand{\mcP}{\mathcal{P}}
\newcommand{\mcQ}{\mathcal{Q}}	\newcommand{\mcR}{\mathcal{R}}
\newcommand{\mcS}{\mathcal{S}}	\newcommand{\mcT}{\mathcal{T}}
\newcommand{\mcU}{\mathcal{U}}	\newcommand{\mcV}{\mathcal{V}}
\newcommand{\mcW}{\mathcal{W}}	\newcommand{\mcX}{\mathcal{X}}
\newcommand{\mcY}{\mathcal{Y}}	\newcommand{\mcZ}{\mathcal{Z}}



%---------------------------------------
% Bold Math Fonts :-
%---------------------------------------

%Captital Letters
\newcommand{\bmA}{\boldsymbol{A}}	\newcommand{\bmB}{\boldsymbol{B}}
\newcommand{\bmC}{\boldsymbol{C}}	\newcommand{\bmD}{\boldsymbol{D}}
\newcommand{\bmE}{\boldsymbol{E}}	\newcommand{\bmF}{\boldsymbol{F}}
\newcommand{\bmG}{\boldsymbol{G}}	\newcommand{\bmH}{\boldsymbol{H}}
\newcommand{\bmI}{\boldsymbol{I}}	\newcommand{\bmJ}{\boldsymbol{J}}
\newcommand{\bmK}{\boldsymbol{K}}	\newcommand{\bmL}{\boldsymbol{L}}
\newcommand{\bmM}{\boldsymbol{M}}	\newcommand{\bmN}{\boldsymbol{N}}
\newcommand{\bmO}{\boldsymbol{O}}	\newcommand{\bmP}{\boldsymbol{P}}
\newcommand{\bmQ}{\boldsymbol{Q}}	\newcommand{\bmR}{\boldsymbol{R}}
\newcommand{\bmS}{\boldsymbol{S}}	\newcommand{\bmT}{\boldsymbol{T}}
\newcommand{\bmU}{\boldsymbol{U}}	\newcommand{\bmV}{\boldsymbol{V}}
\newcommand{\bmW}{\boldsymbol{W}}	\newcommand{\bmX}{\boldsymbol{X}}
\newcommand{\bmY}{\boldsymbol{Y}}	\newcommand{\bmZ}{\boldsymbol{Z}}
%Small Letters
\newcommand{\bma}{\boldsymbol{a}}	\newcommand{\bmb}{\boldsymbol{b}}
\newcommand{\bmc}{\boldsymbol{c}}	\newcommand{\bmd}{\boldsymbol{d}}
\newcommand{\bme}{\boldsymbol{e}}	\newcommand{\bmf}{\boldsymbol{f}}
\newcommand{\bmg}{\boldsymbol{g}}	\newcommand{\bmh}{\boldsymbol{h}}
\newcommand{\bmi}{\boldsymbol{i}}	\newcommand{\bmj}{\boldsymbol{j}}
\newcommand{\bmk}{\boldsymbol{k}}	\newcommand{\bml}{\boldsymbol{l}}
\newcommand{\bmm}{\boldsymbol{m}}	\newcommand{\bmn}{\boldsymbol{n}}
\newcommand{\bmo}{\boldsymbol{o}}	\newcommand{\bmp}{\boldsymbol{p}}
\newcommand{\bmq}{\boldsymbol{q}}	\newcommand{\bmr}{\boldsymbol{r}}
\newcommand{\bms}{\boldsymbol{s}}	\newcommand{\bmt}{\boldsymbol{t}}
\newcommand{\bmu}{\boldsymbol{u}}	\newcommand{\bmv}{\boldsymbol{v}}
\newcommand{\bmw}{\boldsymbol{w}}	\newcommand{\bmx}{\boldsymbol{x}}
\newcommand{\bmy}{\boldsymbol{y}}	\newcommand{\bmz}{\boldsymbol{z}}

%---------------------------------------
% Scr Math Fonts :-
%---------------------------------------

\newcommand{\sA}{{\mathscr{A}}}   \newcommand{\sB}{{\mathscr{B}}}
\newcommand{\sC}{{\mathscr{C}}}   \newcommand{\sD}{{\mathscr{D}}}
\newcommand{\sE}{{\mathscr{E}}}   \newcommand{\sF}{{\mathscr{F}}}
\newcommand{\sG}{{\mathscr{G}}}   \newcommand{\sH}{{\mathscr{H}}}
\newcommand{\sI}{{\mathscr{I}}}   \newcommand{\sJ}{{\mathscr{J}}}
\newcommand{\sK}{{\mathscr{K}}}   \newcommand{\sL}{{\mathscr{L}}}
\newcommand{\sM}{{\mathscr{M}}}   \newcommand{\sN}{{\mathscr{N}}}
\newcommand{\sO}{{\mathscr{O}}}   \newcommand{\sP}{{\mathscr{P}}}
\newcommand{\sQ}{{\mathscr{Q}}}   \newcommand{\sR}{{\mathscr{R}}}
\newcommand{\sS}{{\mathscr{S}}}   \newcommand{\sT}{{\mathscr{T}}}
\newcommand{\sU}{{\mathscr{U}}}   \newcommand{\sV}{{\mathscr{V}}}
\newcommand{\sW}{{\mathscr{W}}}   \newcommand{\sX}{{\mathscr{X}}}
\newcommand{\sY}{{\mathscr{Y}}}   \newcommand{\sZ}{{\mathscr{Z}}}


%---------------------------------------
% Math Fraktur Font
%---------------------------------------

%Captital Letters
\newcommand{\mfA}{\mathfrak{A}}	\newcommand{\mfB}{\mathfrak{B}}
\newcommand{\mfC}{\mathfrak{C}}	\newcommand{\mfD}{\mathfrak{D}}
\newcommand{\mfE}{\mathfrak{E}}	\newcommand{\mfF}{\mathfrak{F}}
\newcommand{\mfG}{\mathfrak{G}}	\newcommand{\mfH}{\mathfrak{H}}
\newcommand{\mfI}{\mathfrak{I}}	\newcommand{\mfJ}{\mathfrak{J}}
\newcommand{\mfK}{\mathfrak{K}}	\newcommand{\mfL}{\mathfrak{L}}
\newcommand{\mfM}{\mathfrak{M}}	\newcommand{\mfN}{\mathfrak{N}}
\newcommand{\mfO}{\mathfrak{O}}	\newcommand{\mfP}{\mathfrak{P}}
\newcommand{\mfQ}{\mathfrak{Q}}	\newcommand{\mfR}{\mathfrak{R}}
\newcommand{\mfS}{\mathfrak{S}}	\newcommand{\mfT}{\mathfrak{T}}
\newcommand{\mfU}{\mathfrak{U}}	\newcommand{\mfV}{\mathfrak{V}}
\newcommand{\mfW}{\mathfrak{W}}	\newcommand{\mfX}{\mathfrak{X}}
\newcommand{\mfY}{\mathfrak{Y}}	\newcommand{\mfZ}{\mathfrak{Z}}
%Small Letters
\newcommand{\mfa}{\mathfrak{a}}	\newcommand{\mfb}{\mathfrak{b}}
\newcommand{\mfc}{\mathfrak{c}}	\newcommand{\mfd}{\mathfrak{d}}
\newcommand{\mfe}{\mathfrak{e}}	\newcommand{\mff}{\mathfrak{f}}
\newcommand{\mfg}{\mathfrak{g}}	\newcommand{\mfh}{\mathfrak{h}}
\newcommand{\mfi}{\mathfrak{i}}	\newcommand{\mfj}{\mathfrak{j}}
\newcommand{\mfk}{\mathfrak{k}}	\newcommand{\mfl}{\mathfrak{l}}
\newcommand{\mfm}{\mathfrak{m}}	\newcommand{\mfn}{\mathfrak{n}}
\newcommand{\mfo}{\mathfrak{o}}	\newcommand{\mfp}{\mathfrak{p}}
\newcommand{\mfq}{\mathfrak{q}}	\newcommand{\mfr}{\mathfrak{r}}
\newcommand{\mfs}{\mathfrak{s}}	\newcommand{\mft}{\mathfrak{t}}
\newcommand{\mfu}{\mathfrak{u}}	\newcommand{\mfv}{\mathfrak{v}}
\newcommand{\mfw}{\mathfrak{w}}	\newcommand{\mfx}{\mathfrak{x}}
\newcommand{\mfy}{\mathfrak{y}}	\newcommand{\mfz}{\mathfrak{z}}

%---------------------------------------
% Bar
%---------------------------------------

%Captital Letters
\newcommand{\barA}{\overline{A}}	\newcommand{\barB}{\overline{B}}
\newcommand{\barC}{\overline{C}}	\newcommand{\barD}{\overline{D}}
\newcommand{\barE}{\overline{E}}	\newcommand{\barF}{\overline{F}}
\newcommand{\barG}{\overline{G}}	\newcommand{\barH}{\overline{H}}
\newcommand{\barI}{\overline{I}}	\newcommand{\barJ}{\overline{J}}
\newcommand{\barK}{\overline{K}}	\newcommand{\barL}{\overline{L}}
\newcommand{\barM}{\overline{M}}	\newcommand{\barN}{\overline{N}}
\newcommand{\barO}{\overline{O}}	\newcommand{\barP}{\overline{P}}
\newcommand{\barQ}{\overline{Q}}	\newcommand{\barR}{\overline{R}}
\newcommand{\barS}{\overline{S}}	\newcommand{\barT}{\overline{T}}
\newcommand{\barU}{\overline{U}}	\newcommand{\barV}{\overline{V}}
\newcommand{\barW}{\overline{W}}	\newcommand{\barX}{\overline{X}}
\newcommand{\barY}{\overline{Y}}	\newcommand{\barZ}{\overline{Z}}
%Small Letters
\newcommand{\bara}{\overline{a}}	\newcommand{\barb}{\overline{b}}
\newcommand{\barc}{\overline{c}}	\newcommand{\bard}{\overline{d}}
\newcommand{\bare}{\overline{e}}	\newcommand{\barf}{\overline{f}}
\newcommand{\barg}{\overline{g}}	\newcommand{\barh}{\overline{h}}
\newcommand{\bari}{\overline{i}}	\newcommand{\barj}{\overline{j}}
\newcommand{\bark}{\overline{k}}	\newcommand{\barl}{\overline{l}}
\newcommand{\barm}{\overline{m}}	\newcommand{\barn}{\overline{n}}
\newcommand{\baro}{\overline{o}}	\newcommand{\barp}{\overline{p}}
\newcommand{\barq}{\overline{q}}	\newcommand{\barr}{\overline{r}}
\newcommand{\bars}{\overline{s}}	\newcommand{\bart}{\overline{t}}
\newcommand{\baru}{\overline{u}}	\newcommand{\barv}{\overline{v}}
\newcommand{\barw}{\overline{w}}	\newcommand{\barx}{\overline{x}}
\newcommand{\bary}{\overline{y}}	\newcommand{\barz}{\overline{z}}

%---------------------------------------
% Greek Letters:-
%---------------------------------------
\newcommand{\eps}{\epsilon}
\newcommand{\veps}{\varepsilon}
\newcommand{\lm}{\lambda}
\newcommand{\Lm}{\Lambda}
\newcommand{\gm}{\gamma}
\newcommand{\Gm}{\Gamma}
\newcommand{\vph}{\varphi}
\newcommand{\ph}{\phi}

\newcommand{\Qed}{\begin{flushright}\qed\end{flushright}}
\newcommand{\parinn}{\setlength{\parindent}{1cm}}
\newcommand{\parinf}{\setlength{\parindent}{0cm}}
\newcommand{\del}[2]{\frac{\partial #1}{\partial #2}}
\newcommand{\Del}[3]{\frac{\partial^{#1} #2}{\partial^{#1} #3}}
\newcommand{\deld}[2]{\dfrac{\partial #1}{\partial #2}}
\newcommand{\Deld}[3]{\dfrac{\partial^{#1} #2}{\partial^{#1} #3}}
\newcommand{\uin}{\mathbin{\rotatebox[origin=c]{90}{$\in$}}}
\newcommand{\usubset}{\mathbin{\rotatebox[origin=c]{90}{$\subset$}}}
\newcommand{\lt}{\left}
\newcommand{\rt}{\right}
\newcommand{\exs}{\exists}
\newcommand{\st}{\strut}
\newcommand{\dps}[1]{\displaystyle{#1}}
\newcommand{\la}{\langle}
\newcommand{\ra}{\rangle}
\newenvironment{solution}
{\textit{\textbf{Solution:}} 
}
{ 
	\hfill $\blacksquare$
	
	\vspace{1cm}
}
\newcommand{\sol}[1]{\begin{solution}#1\end{solution}}
\newcommand{\solve}[1]{\setlength{\parindent}{0cm}\textbf{\textit{Solution: }}\setlength{\parindent}{1cm}#1 \Qed}
\newcommand{\mat}[1]{\left[\begin{matrix}#1\end{matrix}\right]}
\newcommand\numberthis{\addtocounter{equation}{1}\tag{\theequation}}
\newcommand{\handout}[3]{
	\noindent
	\begin{center}
		\framebox{
			\vbox{
				\hbox to 6.5in { {\bf Complexity Theory I } \hfill Jan -- May, 2023 }
				\vspace{4mm}
				\hbox to 6.5in { {\Large \hfill #1  \hfill} }
				\vspace{2mm}
				\hbox to 6.5in { {\em #2 \hfill #3} }
			}
		}
	\end{center}
	\vspace*{4mm}
}

\newcommand{\lecture}[3]{\handout{Lecture #1}{Lecturer: #2}{Scribe:	#3}}
	

\newcommand{\ov}[1]{\overline{#1}}
\newcommand{\thmref}[1]{\hyperref[#1]{Theorem \ref{#1}}}

\setlength{\parindent}{0pt}

%%%%%%%%%%%%%%%%%%%%%%%%%%%%%%%%%%%%%%%%%%%%%%%%%%%%%%%%%%%%%%%%%%%%%%%%%%%%%%%%%%%%%%%%%%%%%%%%%%%%%%%%%%%%%%%%%%%%%%%%%%%%%%%%%%%%%%%%

\begin{document}

%%%%%%%%%%%%%%%%%%%%%%%%%%%%%%%%%%%%%%%%%%%%%%%%%%%%%%%%%%%%%%%%%%%%%%%%%%%%%%%%%%%%%%%%%%%%%%%%%%%%%%%%%%%%%%%%%%%%%%%%%%%%%%%%%%%%%%%%

\textsf{\noindent \large\textbf{Soham Chatterjee} \hfill \textbf{Assignment - 1}\\
    Email: \href{sohamc@cmi.ac.in}{sohamc@email.com} \hfill Roll: BMC202175\\
    \normalsize Course: Algorithmic Coding Theory \hfill Date: September 7, 2023}

%%%%%%%%%%%%%%%%%%%%%%%%%%%%%%%%%%%%%%%%%%%%%%%%%%%%%%%%%%%%%%%%%%%%%%%%%%%%%%%%%%%%%%%%%%%%%%%%%%%%%%%%%%%%%%%%%%%%%%%%%%
% Problem 1
%%%%%%%%%%%%%%%%%%%%%%%%%%%%%%%%%%%%%%%%%%%%%%%%%%%%%%%%%%%%%%%%%%%%%%%%%%%%%%%%%%%%%%%%%%%%%%%%%%%%%%%%%%%%%%%%%%%%%%%%%%

\begin{problem}{%problem statement
		Chapter 1
}{p1
% problem reference text
}
%Problem		
Ex 1.18
\end{problem}

\solve{
%Solution
}


%%%%%%%%%%%%%%%%%%%%%%%%%%%%%%%%%%%%%%%%%%%%%%%%%%%%%%%%%%%%%%%%%%%%%%%%%%%%%%%%%%%%%%%%%%%%%%%%%%%%%%%%%%%%%%%%%%%%%%%%%%
% Problem 2
%%%%%%%%%%%%%%%%%%%%%%%%%%%%%%%%%%%%%%%%%%%%%%%%%%%%%%%%%%%%%%%%%%%%%%%%%%%%%%%%%%%%%%%%%%%%%%%%%%%%%%%%%%%%%%%%%%%%%%%%%%

\begin{problem}{%problem statement
		Chapter 2
}{p2% problem reference text
}
%Problem
Ex 2.13		

\end{problem}

\solve{
%Solution
}
%%%%%%%%%%%%%%%%%%%%%%%%%%%%%%%%%%%%%%%%%%%%%%%%%%%%%%%%%%%%%%%%%%%%%%%%%%%%%%%%%%%%%%%%%%%%%%%%%%%%%%%%%%%%%%%%%%%%%%%%%%
% Problem 3
%%%%%%%%%%%%%%%%%%%%%%%%%%%%%%%%%%%%%%%%%%%%%%%%%%%%%%%%%%%%%%%%%%%%%%%%%%%%%%%%%%%%%%%%%%%%%%%%%%%%%%%%%%%%%%%%%%%%%%%%%%

\begin{problem}{%problem statement
		Chapter 2
	}{p3% problem reference text
	}
	%Problem		
	Ex 2.16
\end{problem}

\solve{
	%Solution
	\begin{enumerate}[label=(\alph*)]
		\item Since $G$ has full rank, $rank(G)=k$. Therefore in the reduced column echelon form of $G$ the first $k$ columns forms a identity matrix $I_k$. We denote the matrix formed by the rest $n-k$ columns by $A$. Since the reduced column echelon form of a matrix and the matrix generate the same vector space they are equivalent. And since the reduced column echelon form can be obtained through the Gaussian elimination method we can convert $G$ to a matrix $G'$ of the form $G'=[I_k|A]$ in polynomial time where $G'$ and $G$ are equivalent.
		\item We should have $GH^T=0$ where $G$ is of the form $G=[I_k|A]$. where $A$ is a $k\times (n-k)$ matrix. Take $H=[-A^T|I_{n-k}]$. Suppose we denote $G=(g_{i,j})_{\substack{1\leq i\leq k\\ 1\leq j\leq n}}$ and $H=(h_{i,j})_{\substack{1\leq i\leq n\\ 1\leq j\leq n-k}}$.  Let $C=GH^T=(c_{i,j})_{\substack{1\leq i\leq k\\ 1\leq j\leq n-k}}$\begin{align*}
			c_{i,j}&=\sum_{m=1}^ng_{i,m}h_{m,j}=\sum_{m=1}^{k}\delta_{i,m}h_{m,j}+\sum_{m=k+1}^ng_{i,m}\delta_{m-k,j}=h_{i,j}+g_{i,k+j}=-a_{i,j}+a_{i,j}=0
		\end{align*}So we get every entry of $C$ is 0. Hence $GH^T=0$. Therefore $H$ is the parity check matrix of $G$ and since $H$ is of the form $H=[-A^T|I_{n-k}]$ so it has full rank $n-k$. Hence $H$ is a parity check matrix.
	\item The general parity check matrix $H$ of the hamming code $[2^r,2^r-1-r,3]$ is the the $i$th column is the binary representation of $i$. Now by gaussian elimination we can convert it to the form $H'=[A\mid I_{r}]$. So now in $H'$ for the last $r$ many columns the $i$th columns is the binary representation of $2^i$. In $H$ the $i$th column for which $2^k<i<2^{k+1}$ in $H'$ it is the $(i-k)$th column. So then the generator matrix of the hamming code $[2^r-1,2^r-1-r,3]$ is the matrix $G=[I_{2^r-1-r}\mid -A^T]$ by part (b)
	\end{enumerate}
}
%%%%%%%%%%%%%%%%%%%%%%%%%%%%%%%%%%%%%%%%%%%%%%%%%%%%%%%%%%%%%%%%%%%%%%%%%%%%%%%%%%%%%%%%%%%%%%%%%%%%%%%%%%%%%%%%%%%%%%%%%%
% Problem 4
%%%%%%%%%%%%%%%%%%%%%%%%%%%%%%%%%%%%%%%%%%%%%%%%%%%%%%%%%%%%%%%%%%%%%%%%%%%%%%%%%%%%%%%%%%%%%%%%%%%%%%%%%%%%%%%%%%%%%%%%%%

\begin{problem}{%problem statement
		Chapter 2
	}{p4% problem reference text
	}
	%Problem		
	Ex 2.17
\end{problem}

\solve{
	%Solution
	\begin{enumerate}[label=(\alph*)]
		\item We encode each alphabet in $(n,k,d)_{2^m}$ in binary $\{0,1\}$. So each alphabet takes $m$ bits to encode. So now in the old code to encode each code in binary we have to encode all the $n$ alphabets in binary which takes total $nm$ bits to encode. So in the new code the code length becomes $nm$. \parinn
		
		Initially $|C|=(2^m)^k=@^{mk}$. Hence the new dimention of the code becomes $km$. And the distance becomes at least the same as old one since we are just encoding all the alphabets in binary. So the new distance $d'\geq d$. The new code is $(nm,km,d'\geq d)_2$.
		\item Like the same logic as for the part (a) we encode all the alphabets in binary which takes $m$ bits. So for each $n$ length old code the new code is of $nm$ length. So the new dimention of the code becomes like before $km$ and the distance is at least $d$. So the new linear code is $[nm,km,d'\geq d]_2$
		\item 
		\item For each $c\in C$ where $C$ is the given linear code $[n,k,d]_q$ we form the new code $c^{\otimes m}\coloneqq\underbrace{c\otimes c\otimes \cdots\otimes c}_{m\text{ times}}$. Let the old alphabet set is $\Sigma$. We create the new alphabet set of size $q^m$ which is the set of all possible $m-tuples$ i.e. $\Sigma' =\lt\{ (q_1,\dots,q_m)\mid q_i\in \Sigma\ \forall \ i\in [m] \rt\}$. So the new alphabet size becomes $|\Sigma'|=q^m$. 
		Now let $c\in C$ is $c=(q_1,\dots,q_n)$. Now if we expand out the $c^{\otimes m}$ each element of it is a $m$-product of the letters from the set $\{q_1,\dots,q_n\}$. So we can represent each element of it as a $m$-tuple. Now each of this tuple is an element of the alphabet set we created just now. So in the new code number of codes remains same but the alphabet size is now $q^m$. Now originally $|C|=q^k=(q^m)^{\frac{k}{m}$ So the dimention is $\frac{k}{m}$. Now for the distance 
	\end{enumerate}
}

%%%%%%%%%%%%%%%%%%%%%%%%%%%%%%%%%%%%%%%%%%%%%%%%%%%%%%%%%%%%%%%%%%%%%%%%%%%%%%%%%%%%%%%%%%%%%%%%%%%%%%%%%%%%%%%%%%%%%%%%%%
% Problem 5
%%%%%%%%%%%%%%%%%%%%%%%%%%%%%%%%%%%%%%%%%%%%%%%%%%%%%%%%%%%%%%%%%%%%%%%%%%%%%%%%%%%%%%%%%%%%%%%%%%%%%%%%%%%%%%%%%%%%%%%%%%

\begin{problem}{%problem statement
		Chapter 5
	}{p1
		% problem reference text
	}
	%Problem
	Ex 5.8		
\end{problem}

\solve{
	%Solution
}


%%%%%%%%%%%%%%%%%%%%%%%%%%%%%%%%%%%%%%%%%%%%%%%%%%%%%%%%%%%%%%%%%%%%%%%%%%%%%%%%%%%%%%%%%%%%%%%%%%%%%%%%%%%%%%%%%%%%%%%%%%
% Problem 6
%%%%%%%%%%%%%%%%%%%%%%%%%%%%%%%%%%%%%%%%%%%%%%%%%%%%%%%%%%%%%%%%%%%%%%%%%%%%%%%%%%%%%%%%%%%%%%%%%%%%%%%%%%%%%%%%%%%%%%%%%%

\begin{problem}{%problem statement
		Chapter 5
	}{p2% problem reference text
	}
	%Problem		
	Ex 5.15
\end{problem}

\solve{
	%Solution
}
%%%%%%%%%%%%%%%%%%%%%%%%%%%%%%%%%%%%%%%%%%%%%%%%%%%%%%%%%%%%%%%%%%%%%%%%%%%%%%%%%%%%%%%%%%%%%%%%%%%%%%%%%%%%%%%%%%%%%%%%%%
% Problem 7
%%%%%%%%%%%%%%%%%%%%%%%%%%%%%%%%%%%%%%%%%%%%%%%%%%%%%%%%%%%%%%%%%%%%%%%%%%%%%%%%%%%%%%%%%%%%%%%%%%%%%%%%%%%%%%%%%%%%%%%%%%

\begin{problem}{%problem statement
		Chapter 5
	}{p3% problem reference text
	}
	%Problem		
	Ex 5.16
\end{problem}

\solve{
	%Solution
	\begin{enumerate}
		\item We have $f(X+Z)=\sum\limits_{i=0}^{t} r_i(X)Z^i$. Now differentiating $f$ with respect to $Z$ we have $$f'(X+Z)=\sum_{i=0}^{t-1}(i+1)r_{i+1}(X)Z^i$$
		Let for $n=k-1$ we have $$f^{(k-1)}(X+Z)=\sum_{i=0}^{t-k+1} \frac{(i+k-1)!}{i!} r_{i+k-1}(X)Z^i$$
		Denote $\dfrac{(i+k-1)!}{i!} r_{i+k-1}(X)=g_{i}(X)$. Then for $n=k$ we have
		\begin{align*}
			f^{(k)}(X+Z) & = \sum_{i=0}^{t-k} (i+1)g_{i+1}Z^i=\sum_{i=0}^{t-k} \frac{((i+1)+k-1)!}{i!} r_{(i+1)+k-1}(X)Z^i\\
			& = \sum_{i=0}^{t-k}\frac{(i+k)!}{i!} r_{i+k}(X)Z^i
		\end{align*}
		Hence by mathematical induction we have $$f^{(n)}(X+Z)=\sum_{i=0}^{t-n}\frac{(i+n)!}{i!} r_{i+n}(X)Z^i$$
		Therefore $$f^{(n)}(X)=f^{(n)}(X+0)=\sum_{i=0}^{t-n}\frac{(i+n)!}{i!} r_{i+n}(X)0^i=\frac{n!}{0!}r_{n}(X)=n!r_n(X)$$
		\item Let $char(\bbF_q)=m$. So $j\geq m$. Hence $j!=j(j-1)\cdots (m+1)m(m-1)!=mk$ where $k=j(j-1)\cdots (m+1)(m-1)!$. Since $f^{(j)}(X)=j!r_j(X)=m\lt( kr_j(X) \rt)\equiv 0$.
	\end{enumerate}
}
%%%%%%%%%%%%%%%%%%%%%%%%%%%%%%%%%%%%%%%%%%%%%%%%%%%%%%%%%%%%%%%%%%%%%%%%%%%%%%%%%%%%%%%%%%%%%%%%%%%%%%%%%%%%%%%%%%%%%%%%%%
% Problem 8
%%%%%%%%%%%%%%%%%%%%%%%%%%%%%%%%%%%%%%%%%%%%%%%%%%%%%%%%%%%%%%%%%%%%%%%%%%%%%%%%%%%%%%%%%%%%%%%%%%%%%%%%%%%%%%%%%%%%%%%%%%

\begin{problem}{%problem statement
		Chapter 5
	}{p4% problem reference text
	}
	%Problem		
	Ex 5.17
\end{problem}

\solve{
	%Solution
}
\end{document}
