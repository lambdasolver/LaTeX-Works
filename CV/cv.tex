%% start of file `template.tex'.
%% Copyright 2006-2013 Xavier Danaux (xdanaux@gmail.com).
%
% This work may be distributed and/or modified under the
% conditions of the LaTeX Project Public License version 1.3c,
% available at http://www.latex-project.org/lppl/.


\documentclass[10pt,a4paper,sans,colorlinks]{moderncv}        % possible options include font size ('10pt', '11pt' and '12pt'), paper size ('a4paper', 'letterpaper', 'a5paper', 'legalpaper', 'executivepaper' and 'landscape') and font family ('sans' and 'roman')

% modern themes
\moderncvstyle{banking}                            % style options are 'casual' (default), 'classic', 'oldstyle' and 'banking'
\moderncvcolor{blue}
                                % color options 'blue' (default), 'orange', 'green', 'red', 'purple', 'grey' and 'black'
\AtBeginDocument{
	\hypersetup{colorlinks,urlcolor=myblue}
}                  
\definecolor{myblue}{HTML}{3972b3}                                          
%\renewcommand{\familydefault}{\sfdefault}         % to set the default font; use '\sfdefault' for the default sans serif font, '\rmdefault' for the default roman one, or any tex font name
\nopagenumbers{}                                  % uncomment to suppress automatic page numbering for CVs longer than one page
% character encoding
\usepackage[utf8]{inputenc}                       % if you are not using xelatex ou lualatex, replace by the encoding you are using
%\usepackage{CJKutf8}                              % if you need to use CJK to typeset your resume in Chinese, Japanese or Korean
\usepackage{amsmath}

\usepackage{multicol}

% adjust the page margins
%\usepackage[scale=0.8]{geometry}
\usepackage[left=2cm, right=2cm, top=1cm, bottom=0.7cm]{geometry}
%\setlength{\hintscolumnwidth}{3cm}                % if you want to change the width of the column with the dates
%\setlength{\makecvheadnamewidth}{10cm}           % for the 'classic' style, if you want to force the width allocated to your name and avoid line breaks. be careful though, the length is normally calculated to avoid any overlap with your personal info; use this at your own typographical risks...

\usepackage{import}

% personal data
\name{Soham}{Chatterjee} 

%\address{28, Rue Marcelin Berthelot, 92120 - Montrouge - France}% optional, remove / comment the line if not wanted; the "postcode city" and and "country" arguments can be omitted or provided empty
%\phone[mobile]{+919433548242}      % optional, remove / comment the line if not wanted
%\phone[fixed]{01234 123456}                    % optional, remove / comment the line if not wanted
%\phone[fax]{+3~(456)~789~012}                      % optional, remove / comment the line if not wanted
\email{sohamc@cmi.ac.in / sohamchatterjee999@gmail.com}                               % optional, remove / comment the line if not wanted
\homepage{sohamch08.github.io}                         % optional, remove / comment the line if not wanted

%\extrainfo{Date of Birth: 8th September, 2002}                % optional, remove / comment the line if not wanted
%\photo[64pt][0.4pt]{me.png}                       % optional, remove / comment the line if not wanted; '64pt' is the height the picture must be resized to, 0.4pt is the thickness of the frame around it (put it to 0pt for no frame) and 'picture' is the name of the picture file
%\quote{Some quote}                                 % optional, remove / comment the line if not wanted

% to show numerical labels in the bibliography (default is to show no labels); only useful if you make citations in your resume
%\makeatletter
%\renewcommand*{\bibliographyitemlabel}{\@biblabel{\arabic{enumiv}}}
%\makeatother
%\renewcommand*{\bibliographyitemlabel}{[\arabic{enumiv}]}% CONSIDER REPLACING THE ABOVE BY THIS
\newcommand{\itemTitle}[1]{\vspace*{1pt}
	\item[] \underline{#1}\vspace{2pt}
}
\newcommand{\heading}[2]{
	\hspace{0pt}#1\hfill#2\\
}

% Adds \textbf to \heading
\newcommand{\headingBf}[2]{
	\heading{\textbf{#1}}{\textbf{#2}}
}

% Adds \textit to \heading
\newcommand{\headingIt}[2]{
	\heading{\textit{#1}}{\textit{#2}}
}

% Template for itemized lists
% Usage: \begin{resume_list} [items] \end{resume_list}

% bibliography with mutiple entries
%\usepackage{multibib}
%\newcites{book,misc}{{Books},{Others}}
%----------------------------------------------------------------------------------
%            content
%----------------------------------------------------------------------------------
\begin{document}
%\begin{CJK*}{UTF8}{gbsn}                          % to typeset your resume in Chinese using CJK
%-----       resume       ---------------------------------------------------------
\makecvtitle
\vspace{-30pt}
%{ \textbf{Expertise}: Marketing online et offline, Coordination des projets, Branding, Merchandising.\\
%\textbf{Projets réalisés}:http://gnt.globo.com/especiais/projetos-multitelas}

\section{Education}



\begin{itemize}
	\item{\cventry{2021 -- Ongoing}{B. Sc - Mathematics and Computer Science}{Chennai Mathematical Institute}{Chennai, Tamilnadu, India}{}{
	            }}
	\vspace{4pt}
	\vspace{4pt}
	\item{\cventry{2018 -- 2020}{Higher Secondary ($12^{\text{th}}$ Standard) Education}{Baranagar Narendranath Vidyamandir}{Kolkata, West Bengal, India}{}{
	            }}

	\vspace{4pt}

	\item{\cventry{2008 -- 2018}{Secondary ($10^{\text{th}}$ Standard) Education}{Baranagar Ramakrishna Mission Ashrama High School}{Kolkata, West Bengal, India}{}{
	            }}


\end{itemize}

\section{Academic Achievements}



\begin{itemize}
	\item \textbf{GS Exam, I-PhD, Computer Science, 2024\hfill TIFR, Mumbai, India}
	
	Nation wide entrance exam in Computer Science for Tata Insititute of Fundamental Research. Only 2 people got selected.
	\item \textbf{JEST, I-PhD, Theoretical Computer Science, 2024 - Rank 5\hfill IMSC, India}
	
	Nation wide entrance exam in Computer Science for Insitute of Mathematical Sciences

	\item \textbf{NEST, B.Sc., 2021\hfill NISER, India}
	
	Nation wide bachelors entrance exam for National Institute of Science Education and Research

	\item \textbf{WBJEE, B.Tech, 2020 - Rank 1893\hfill WBJEEB, India}
	
	Joint Entrance exam for B.Tech for West Bengal state
	\item \textbf{12$^{th}$ Statistics Olympiad, 2020 - Rank 28\hfill AIMSCS, India}
	
	Organized by C R Rao Advanced Institute of Mathematics, Statistics and Computer Science
\end{itemize}

%%%%%%%%%%%%%%%%%%%%%%%%%%%%%%%%%%%%%%%%%%%%%%%%%%
%%%%%%%%%%%% Internships %%%%%%%%%%%%%%%%%%%%%%%%%
%%%%%%%%%%%%%%%%%%%%%%%%%%%%%%%%%%%%%%%%%%%%%%%%%%


\section{Internships}
\begin{itemize}
	\item \headingBf{Polyhedral Combinatorics and Derandomization of Isolation Lemma}{}
	\headingIt{Supervisor: \href{https://www.cse.iitb.ac.in/~rgurjar/}{Rohit Gurjar}, IIT Mumbai}{May - Jul, 2024}
	\begin{itemize}
		\item I read the papers\begin{itemize}
			\item  'Bipartite Perfect Matching is in \textsc{Quasi-NC}' by Fenner, Gurjar and Thierauf
			\item 'Linear Matroid Intersection Is in \textsc{Quasi-NC}' by Gurjar and Thierauf
			\item `Fractional Linear Matroid Matching is in \textsc{Quasi-NC}' by Gurjar, Oki and Raj
		\end{itemize}Learned how the idea of giving nonzero circulations to cycles and bounding number of integral vectors (corresponding those cycles) twice the size of smallest vector helps construct an isolating weights for bipartite perfect matching polytope to fractional matroid matching polytopes
		\item Additionally I read about isolating a path connecting the source vertex and sink vertex in a black-box layered graph from the paper `Derandomizing Isolation in Space-Bounded Settings' by Melkebeek and Prakriya.
	\end{itemize}
	
	\item \headingBf{Quantum Property Testing of Junta Functions and Partially Symmetric Functions.}{}
	\headingIt{Supervisor: \href{https://sites.google.com/site/homepagearijitghosh/}{Arijit Ghosh}, Indian Statistical Institute, Kolkata}{Dec, 2024 -- Going on}
	\begin{itemize}
		\item I learned about Fourier Analysis of Quantum Boolean Functions and Quantum algorithms for Testing and Learning Stabilizer States from Quantum boolean functions' by Montanaro and Osborne 
		\item Also  learned about Classical Junta Testing from Eric Blais' paper Testing Juntas Nearly Optimally and then read about Quantum Junta Testing Algorithm from `Testing and Learning Quantum Juntas Nearly Optimally' by Chen, Nadimpalli and Yuen
		\item And I learned about Partially Symmetric Boolean Functions and it's classical algorithm of testing partially symmetric functions from the paper `Partially Symmetric Functions are Efficiently Isomorphism-Testable' by Blais, Weinstein and Yoshida and we were trying to come up with a Quantum Algorithm for Testing Partially Symmetric Boolean Functions.
	\end{itemize}
	
	
	\item \headingBf{Factorization of Arithmetic Circuits in Algebraic Complexity Theory}{}
	\headingIt{Supervisor: \href{https://www.cse.iitk.ac.in/users/nitin/}{Nitin Saxena}, IIT Kanpur}{May - Jul, 2022}
	\begin{itemize}
		\item I read `Discovering the roots: Uniform closure results for algebraic classes under factoring' by Dutta, Saxena and Sinhababu  where I learned factorizing multivariate arithmetic circuits and \textsc{VP} closure under factorization. 
		\item  Also read the Kaltofen's proof of \textsc{VP} closed under factorization.
		\item Also learned how Polynomial Identity Testing and Multivariate Factorizations are equivalent from `Equivalence of Polynomial Identity Testing and Deterministic Multivariate Polynomial Factorization' by  Kopparty, Saraf and Shpilka
		\item I also read how \textsc{VBP} is closed under factorization from  Sinhababu and Tierauf's paper `Factorization of Polynomials given by Arithmetic Branching Programs'
		\item Learned about the difficulties about proving factor closure for \textsc{VF} from the above mentioned two papers. I read about factorization of formulas with individual degree bounded form the paper `Factors of low individual degree polynomials' by Rafael Oliveira and we were trying to remove the condition for formulas
	\end{itemize}
	
	
	\item \headingBf{Computational Number Theory and Algebra for Algebraic Comlexity Theory.}{}
	\headingIt{Supervisor: \href{https://www.cse.iitk.ac.in/users/nitin/}{Nitin Saxena}, IIT Kanpur}{Dec - Jan, 2022}
	\begin{itemize}
		\item I learned about Computational Number Theory and Algebra from Nitin Saxena's Course and read the book `Modern Computer Algebra' by Von Zur Gathen and Jurgen Gerhard
		\item Also I learned about Arithmetic Circuits from \href{https://www.nowpublishers.com/article/Details/TCS-039}{Amir Shpilka's Survey} and 
		\href{https://github.com/dasarpmar/lowerbounds-survey}{Ramprasad Saptharishi's Survey} on Arithmetic Circuits.
	\end{itemize}
\end{itemize}
%%%%%%%%%%%%%%%%%%%%%%%%%%%%%%%%%%%%%%%%%%%%%%%%%%
%%%%%%%% Course Projects %%%%%%%%%%%%%%%%%%%%%%%%%
%%%%%%%%%%%%%%%%%%%%%%%%%%%%%%%%%%%%%%%%%%%%%%%%%%

\section{Course Projects}
\begin{itemize}
	\item \textbf{Presentation on Hensel Lifting, Newton Iteration and Factorization in Valuation Rings}

	\hfill  Presented the paper ``\href{https://www.ams.org/journals/mcom/1984-42-166/S0025-5718-1984-0736459-9/}{Hensel and Newton methods in valuation rings}” by J von zur Gathen in Algebra and Computation course at CMI.
	\item \textbf{Presentation on Iterated Mod Problem\hfill \href{https://sohamch08.github.io/assets/parallel-presentation-iterated-mod.pdf}{Slides}}
	
	\hfill  Presented the paper ``\href{https://www.sciencedirect.com/science/article/pii/0890540189900084}{Iterated Mod Problem}” by Karloff and Ruzzo in Parallel 	Algorithm and Complexity course at CMI.
	\item \textbf{Report on Algebraic Geometric Codes\hfill \href{https://sohamch08.github.io//assets/act-report.pdf}{Link}}
	
	\hfill Followed the \href{https://dl.acm.org/doi/abs/10.5555/334156.334207}{Survey} by Blake, Heegard, Høholdt, and Wei and Gil Cohen's \href{https://www.gilcohen.org/2022-ag-codes}{Course}  in Algorithmic Coding Theory (II) course at CMI.
	\item \textbf{Qiskit Implementation of Quantum Circuit of Modular Exponentiation\hfill \href{https://github.com/bluecheese123/-Best_Project-}{Link}}
	
	\hfill Implemented the paper: ``\href{https://arxiv.org/pdf/quant-ph/9511018.pdf}{Quantum Networks for Elementary Arithmetic Operations}" by  Vedral, Barenco and Artur Ekert 
	
	\item \textbf{Qiskit Implementation of Kushlevitz and Mansour Algorithm}\hfill \href{https://github.com/sohamch08/Qiskit-Quantum-Algo/blob/master/Kushlevitz%20and%20Mansour%20Algorithm.ipynb}{\textbf{Link}}
	
	\hfill Implemented the paper: ``\href{https://dl.acm.org/doi/pdf/10.1145/103418.103466}{Learning Decision Trees Using The Fourier Spectrum}” by Kushilevitz and Mansour
	
	\item \textbf{Qiskit Implementation of Some Quantum Algorithms\hfill  \href{https://github.com/sohamch08/Qiskit-Quantum-Algo}{Link}}
	
	\hfill Implemented Grover Search for 2 × 2 sudoku and Iterative Phase Estimation
\end{itemize}
\section{Workshop, Lecture Serires Attended}
\begin{itemize}
	\item{\cventry{2024, Jan-May}{Chennai, India}{\href{https://www.cmi.ac.in/activities/kohli-centre/quantum-semester-2024/}{Quantum Semester Online}}{Chennai Mathematical Institute}{}{
	}}
	\vspace{4pt}
	
	\item{\cventry{September, 2023}{Chennai, India}{\href{https://www.cmi.ac.in/~mkummini/sagedays122/index.html}{Sage Days 122}}{Chennai Mathematical Institute}{}{
	}}
	\vspace{4pt}
	
	\item{\cventry{Online: August, 2023}{Mumbai, India}{\href{https://sohamch08.github.io/assets/certificates_merged_SohamChatterjee.pdf}{p-adic Number Theory Lecture Series: Ram Murty}}{Math Dept, University of Mumbai}{}{
	}}
	\vspace{4pt}	
\end{itemize}

\section{Relevent Courses}



\begin{multicols}{2}
	\textbf{Math Courses:}{}
	\begin{itemize}
		\itemTitle{Algebra}
		\item Linear Algebra (Algebra 1)
		\item Group Theory (Algebra 2)
		\item Ring and  Field Theory (Algebra 3)
		\item Commutative Algebra
		\itemTitle{Analysis}
		\item Real Analysis (Analysis 1)
		\item Analysis in Euclidean Space (Analysis 2)
		\item Analysis in Metric Space (Analysis 3)
		\vspace{3pt}
		\itemTitle{Other Math Courses}
		\item Complex Analysis
		\item Calculus
		\item Probability Theory
		\item Topology
	\end{itemize}
	\columnbreak
	
	\textbf{Computer Science Courses:}{}
	\begin{itemize}
		\item Discrete Mathematics
		\item  Design and Analysis of Algorithms
		\item  Theory of Computation
		\item Complexity Theory
		\item Parallel Algorithms and Complexity
		\item  Expander Graphs and Application
		\item  Algorithmic Coding Theory (Two Parts)
		\item  Quantum Algorithmic Thinking
		\item  Quantum Information Theory
	\end{itemize}
\end{multicols}

\section{Computer Skills}

\begin{itemize}

	\item \textbf{Programming Languages:} C (Basic), Python (Intermediate), Qiskit (Intermediate), Haskell (Basic), Java (Basic), Unix/Linux Shell Scripting, HTML, CSS

	\item \textbf{Technical Skills:} \LaTeX (Advanced), Markdown, Git, Basic works in terminal, VIM, Obsidian


\end{itemize}
%\section{Hobbies}
%
%\begin{itemize}
%
%	\item  Tinkering \LaTeX, Watch Anime, Listen Music (J-pop, Western), Theming linux desktop
%
%
%\end{itemize}
% Publications from a BibTeX file without multibib
%  for numerical labels: \renewcommand{\bibliographyitemlabel}{\@biblabel{\arabic{enumiv}}}% CONSIDER MERGING WITH PREAMBLE PART
%  to redefine the heading string ("Publications"): \renewcommand{\refname}{Articles}
%	\nocite{*}
%	\bibliographystyle{plain}
%	\bibliography{publications}                        % 'publications' is the name of a BibTeX file

% Publications from a BibTeX file using the multibib package
%\section{Publications}
%\nocitebook{book1,book2}
%\bibliographystylebook{plain}
%\bibliographybook{publications}                   % 'publications' is the name of a BibTeX file
%\nocitemisc{misc1,misc2,misc3}
%\bibliographystylemisc{plain}
%\bibliographymisc{publications}                   % 'publications' is the name of a BibTeX file

%-----       letter       ---------------------------------------------------------

\end{document}


%% end of file `template.tex'.
