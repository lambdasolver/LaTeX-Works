%% start of file `template.tex'.
%% Copyright 2006-2013 Xavier Danaux (xdanaux@gmail.com).
%
% This work may be distributed and/or modified under the
% conditions of the LaTeX Project Public License version 1.3c,
% available at http://www.latex-project.org/lppl/.


\documentclass[10pt,a4paper,sans,colorlinks]{moderncv}        % possible options include font size ('10pt', '11pt' and '12pt'), paper size ('a4paper', 'letterpaper', 'a5paper', 'legalpaper', 'executivepaper' and 'landscape') and font family ('sans' and 'roman')

% modern themes
\moderncvstyle{banking}                            % style options are 'casual' (default), 'classic', 'oldstyle' and 'banking'
\moderncvcolor{blue}
                                % color options 'blue' (default), 'orange', 'green', 'red', 'purple', 'grey' and 'black'
\AtBeginDocument{
	\hypersetup{colorlinks,urlcolor=myblue}
}                  
\definecolor{myblue}{HTML}{3972b3}                                          
%\renewcommand{\familydefault}{\sfdefault}         % to set the default font; use '\sfdefault' for the default sans serif font, '\rmdefault' for the default roman one, or any tex font name
\nopagenumbers{}                                  % uncomment to suppress automatic page numbering for CVs longer than one page
% character encoding
\usepackage[utf8]{inputenc}                       % if you are not using xelatex ou lualatex, replace by the encoding you are using
%\usepackage{CJKutf8}                              % if you need to use CJK to typeset your resume in Chinese, Japanese or Korean
\usepackage{amsmath}

\usepackage{multicol}

% adjust the page margins
%\usepackage[scale=0.8]{geometry}
\usepackage[left=2cm, right=2cm, top=1cm, bottom=0.7cm]{geometry}
%\setlength{\hintscolumnwidth}{3cm}                % if you want to change the width of the column with the dates
%\setlength{\makecvheadnamewidth}{10cm}           % for the 'classic' style, if you want to force the width allocated to your name and avoid line breaks. be careful though, the length is normally calculated to avoid any overlap with your personal info; use this at your own typographical risks...

\usepackage{import}

% personal data
\name{Soham}{Chatterjee} 

%\address{28, Rue Marcelin Berthelot, 92120 - Montrouge - France}% optional, remove / comment the line if not wanted; the "postcode city" and and "country" arguments can be omitted or provided empty
%\phone[mobile]{+919433548242}      % optional, remove / comment the line if not wanted
%\phone[fixed]{01234 123456}                    % optional, remove / comment the line if not wanted
%\phone[fax]{+3~(456)~789~012}                      % optional, remove / comment the line if not wanted
\email{sohamc@cmi.ac.in / sohamchatterjee999@gmail.com}                               % optional, remove / comment the line if not wanted
\homepage{sohamch08.github.io}                         % optional, remove / comment the line if not wanted

%\extrainfo{Date of Birth: 8th September, 2002}                % optional, remove / comment the line if not wanted
%\photo[64pt][0.4pt]{me.png}                       % optional, remove / comment the line if not wanted; '64pt' is the height the picture must be resized to, 0.4pt is the thickness of the frame around it (put it to 0pt for no frame) and 'picture' is the name of the picture file
%\quote{Some quote}                                 % optional, remove / comment the line if not wanted

% to show numerical labels in the bibliography (default is to show no labels); only useful if you make citations in your resume
%\makeatletter
%\renewcommand*{\bibliographyitemlabel}{\@biblabel{\arabic{enumiv}}}
%\makeatother
%\renewcommand*{\bibliographyitemlabel}{[\arabic{enumiv}]}% CONSIDER REPLACING THE ABOVE BY THIS

% bibliography with mutiple entries
%\usepackage{multibib}
%\newcites{book,misc}{{Books},{Others}}
%----------------------------------------------------------------------------------
%            content
%----------------------------------------------------------------------------------
\begin{document}
%\begin{CJK*}{UTF8}{gbsn}                          % to typeset your resume in Chinese using CJK
%-----       resume       ---------------------------------------------------------
\makecvtitle
\vspace{-30pt}
%{ \textbf{Expertise}: Marketing online et offline, Coordination des projets, Branding, Merchandising.\\
%\textbf{Projets réalisés}:http://gnt.globo.com/especiais/projetos-multitelas}

\section{Education}



\begin{itemize}
	\item{\cventry{2021 -- Ongoing}{B. Sc - Mathematics and Computer Science}{Chennai Mathematical Institute}{Chennai, Tamilnadu, India}{}{
	            }}
	\vspace{4pt}
	\item{\cventry{2020 -- 2021}{B. Tech 1st Year - Electronics and Communication Engineering}{University of Calcutta}{Kolkata, West Bengal, India}{}{
	            }}

	\vspace{4pt}
	\item{\cventry{2018 -- 2020}{Higher Secondary ($12^{\text{th}}$ Standard)}{Baranagar Narendranath Vidyamandir}{Kolkata, West Bengal, India}{}{
	            }}

	\vspace{4pt}

	\item{\cventry{2008 -- 2018}{Secondary ($10^{\text{th}}$ Standard)}{Baranagar Ramakrishna Mission Ashrama High School}{Kolkata, West Bengal, India}{}{
	            }}


\end{itemize}

\section{Academic Achievements}



\begin{itemize}

	\item{\cventry{2021}{Entrance exam of Chennai Mathematical Institute}{CMI Entrance}{Chennai Mathematical Institute}{}{}}

	\item{\cventry{2021}{Entrance exam of National Institute of Science Education and Research (NISER)}{NEST}{NISER}{}{}}

	\item{\cventry{2020}{West Bengal Joint Entrance Exam}{WBJEE - Rank 1893}{WBJEEB}{}{}}
	\item{\cventry{2020}{C R Rao Advanced Institute of Mathematics, Statistics and Computer Science (AIMSCS)}{12th Statistics Olympiad - Rank 108}{AIMSCS}{}{}}

\end{itemize}

\section{Internship}
\begin{itemize}
	\item \textbf{Ramanujan's work on theta functions and $q$-series and their connections with number theory.}

	      \hfill Under Professor \href{https://www.iitg.ac.in/rupam/}{Rupam Barman}, IIT Guahati during the summer break in May -- Jul, 2022.

	\item \textbf{Computational Number Theroy and Algebra for Algebraic Comlexity Theory }

	      \hfill Under Professor \href{https://www.cse.iitk.ac.in/users/nitin/}{Nitin Saxena}, IIT Kanpur during the winter break in Dec -- Jan, 2022.
	      
	\item \textbf{Factorization of Formula Arithmetic Circuits in Algebraic Complexity Theory }
	
	\hfill Under Professor \href{https://www.cse.iitk.ac.in/users/nitin/}{Nitin Saxena}, IIT Kanpur during the summer break in May -- July, 2023.
	\item \textbf{Quantum Property Testing and Junta Functions and Partially Symmetric Functions.}
	
	\hfill Under Professor \href{https://sites.google.com/site/homepagearijitghosh/}{Arijit Ghosh}, ISI Kolkata during the winter break in Dec -- Jan, 2023.
\end{itemize}
\section{Course Projects}
\begin{itemize}
	\item \textbf{Presentation on Iterated Mod Problem:: \href{https://sohamch08.github.io/files/parallel-presentation-iterated-mod.pdf}{Slides}}
	
	\hfill  Presented the paper ``\href{https://www.sciencedirect.com/science/article/pii/0890540189900084}{Iterated Mod Problem}” by Howard J. Karloff and Walter L. Ruzzo in Parallel 	Algorithm and Complexity course.
	\item \textbf{Report on Algebraic Geometric Codes: \href{https://sohamch08.github.io//files/act-report.pdf}{Link}}
	
	\hfill Followed the \href{https://dl.acm.org/doi/abs/10.5555/334156.334207}{Survey} by Ian Blake, Chris Heegard, Tom Høholdt, and Victor Wei and Gil Cohen's \href{https://www.gilcohen.org/2022-ag-codes}{Course} 
	\item \textbf{Qiskit Implementation of Quantum Circuit of Modular Exponentiation: \href{https://github.com/bluecheese123/-Best_Project-}{Link}}
	
	\hfill Implemented the paper: ``\href{https://arxiv.org/pdf/quant-ph/9511018.pdf}{Quantum Networks for Elementary Arithmetic Operations}" by Vlatko Vedral, Adriano Barenco and Artur Ekert 
	
	\item \textbf{Qiskit Implementation of Kushlevitz and Mansour Algorithm:} \href{https://github.com/sohamch08/Qiskit-Quantum-Algo/blob/master/Kushlevitz%20and%20Mansour%20Algorithm.ipynb}{\textbf{Link}}
	
	\hfill Implemented the paper: ``\href{https://dl.acm.org/doi/pdf/10.1145/103418.103466}{Learning Decision Trees Using The Fourier Spectrum}” by Eyal Kushilevitz and Yishay Mansour
	
	\item \textbf{Qiskit Implementation of Some Quantum Algorithms: \href{https://github.com/sohamch08/Qiskit-Quantum-Algo}{Link}}
	
	\hfill Implemented Grover Search for 2 × 2 sudoku and Iterative Phase Estimation
\end{itemize}
\section{Workshop, Lecture Serires Attended}
\begin{itemize}
	\item{\cventry{Currently going on: 2024, Jan-May}{Chennai, India}{\href{https://www.cmi.ac.in/activities/kohli-centre/quantum-semester-2024/}{Quantum Semester Online}}{Chennai Mathematical Institute}{}{
	}}
	\vspace{4pt}
	
	\item{\cventry{September, 2023}{Chennai, India}{\href{https://www.cmi.ac.in/~mkummini/sagedays122/index.html}{Sage Days 122}}{Chennai Mathematical Institute}{}{
	}}
	\vspace{4pt}
	
	\item{\cventry{Online: August, 2023}{Mumbai, India}{\href{https://sohamch08.github.io/files/certificates_merged_SohamChatterjee.pdf}{p-adic Number Theory Lecture Series: Ram Murty}}{Math Dept, University of Mumbai}{}{
	}}
	\vspace{4pt}	
\end{itemize}

\section{Topics I Learned}



\begin{itemize}
	\item 	\textbf{Math Topics:-}
 \begin{multicols}{2}
			      	\begin{itemize}
			      		\item Real Analysis
			      		\item Analysis over Euclidean Space
			      		\item Analysis over Metric Space
			      		 \item {Complex Analysis}
			      		  \item {Probability Theory}
			      		 
			      		 
			      		 \item {Calculus}
			      		 \item {Differential Equations}
			      \item {General Topology}
			       \item {Algebraic Topology (Introductory)}
			     
			     
			     \columnbreak
			     \item Linear Algebra
			     \item Group Theory
			     \item Ring Theory
			     \item Field Theory 
			     \item Galois Theory
			     \item {Commutative Algebra}
			     \item {Algebraic Curves}
			   
			      
	      \end{itemize}
			      \end{multicols}
	\item \textbf{Computer Science Topics:-}
	      \begin{itemize}


\item \textbf{Theoretical Computer Science Topics:} \begin{itemize}
\item Design and Analysis of Algorithms - \href{https://www.cmi.ac.in/~gphilip/}{Geevarghese Philip} and \href{https://www.cmi.ac.in/~sdatta/}{Samir Dutta}
\item Theory of Computation - \href{https://www.cmi.ac.in/~kumar/}{Narayan Kumar} and \href{https://www.cmi.ac.in/~aiswarya/}{C. Aiswarya}
\item Complexity Theory - \href{https://www.cmi.ac.in/~partham/}{Partha Mukhopadhyay}
\item Expander Graphs and Application - \href{https://www.cmi.ac.in/~partham/}{Partha Mukhopadhyay} - (Attending)
\item Higher Dimensional Expanders (Paper: \href{https://arxiv.org/abs/1811.01816}{Log Concave Polynomial 2}) - \href{https://www.cmi.ac.in/~partham/}{Partha Mukhopadhyay} - (Attending)
\item Parallel Algorithms and Complexity - \href{https://www.cmi.ac.in/~sdatta/}{Samir Dutta}
\item Algorithmic Coding Theory - \href{https://www.cmi.ac.in/people/fac-profile.php?id=amitks}{Amit Kumar Sinhababu}
\item Algebra and Computation - \href{https://www.cmi.ac.in/people/fac-profile.php?id=amitks}{Amit Kumar Sinhababu} and \href{https://sites.google.com/view/sumghosh/home}{Sumanta Ghosh} - (Attending)
\item Quantum Algorithmic Thinking - \href{https://www.cmi.ac.in/~partham/}{Partha Mukhopadhyay}
\item Classical and Quantum Information Theory - Arun Padakandla - (Attending)
\item Discrete Mathematics - \href{https://sites.google.com/cse.iitm.ac.in/c-ramya}{C Ramya} \& \href{https://www.cmi.ac.in/~partham/}{Partha Mukhopadhyay}
\item Arithmetic Circuits - \href{https://www.cse.iitk.ac.in/users/nitin/}{Nitin Saxena}
\item Computational Algebra and Number Theory - \href{https://www.cse.iitk.ac.in/users/nitin/}{Nitin Saxena}
\item Lambda Calculus
\item Introductory Concurrent Programming
		            \end{itemize}
		            \vspace*{5mm}
		      \item \textbf{Other CS Topics:} \begin{itemize}
			            \item Introduction to Functional Programming (Haskell)
			            \item Advanced Programming with Python - \href{https://www.cmi.ac.in/~sdatta/}{Samir Dutta}
			            \item Programming Language Concepts using Java
		            \end{itemize}

	      \end{itemize}

\end{itemize}



\section{Computer Skills}

\begin{itemize}

	\item \textbf{Programming Languages:} C (Basic), Python (Intermediate), Haskell (Basic), Java (Intermediate), Unix/Linux Shell Scripting, HTML, CSS

	\item \textbf{Technical Skills:} \LaTeX (Advanced), Markdown, Git, Basic works in terminal, VIM, Obsidian


\end{itemize}
%\section{Hobbies}
%
%\begin{itemize}
%
%	\item  Tinkering \LaTeX, Watch Anime, Listen Music (J-pop, Western), Theming linux desktop
%
%
%\end{itemize}
% Publications from a BibTeX file without multibib
%  for numerical labels: \renewcommand{\bibliographyitemlabel}{\@biblabel{\arabic{enumiv}}}% CONSIDER MERGING WITH PREAMBLE PART
%  to redefine the heading string ("Publications"): \renewcommand{\refname}{Articles}
%	\nocite{*}
%	\bibliographystyle{plain}
%	\bibliography{publications}                        % 'publications' is the name of a BibTeX file

% Publications from a BibTeX file using the multibib package
%\section{Publications}
%\nocitebook{book1,book2}
%\bibliographystylebook{plain}
%\bibliographybook{publications}                   % 'publications' is the name of a BibTeX file
%\nocitemisc{misc1,misc2,misc3}
%\bibliographystylemisc{plain}
%\bibliographymisc{publications}                   % 'publications' is the name of a BibTeX file

%-----       letter       ---------------------------------------------------------

\end{document}


%% end of file `template.tex'.
